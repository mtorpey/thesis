\chapter{Introduction}
\label{chap:intro}

Things to define/explain:

\begin{itemize}
\item Lattices of congruences (intersection, join, etc.)
\end{itemize}

\begin{definition}
  \label{def:semigroup}
  A \textbf{semigroup} is a set $S$ together with
  a binary operation $*: S \times S \to S$ such that
  $$(x * y) * z = x * (y * z),$$
  for all $x, y, z \in S$.
\end{definition}

\begin{definition}
  \label{def:congruence}
  Let $S$ be a semigroup, and let $\rho$ be an equivalence relation on $S$.  The
  relation $\rho$ is:
  \begin{itemize}
  \item a \textbf{left congruence} if $(x, y) \in \rho$ implies that
    $(ax, ay) \in \rho$ for all $a \in S$;
  \item a \textbf{right congruence} if $(x, y) \in \rho$ implies that
    $(xa, ya) \in \rho$ for all $a \in S$;
  \item a \textbf{two-sided congruence} if it is both a left congruence and a
    right congruence.
  \end{itemize}
\end{definition}

If we talk about a \textit{congruence} without specifying that it is left or
right, it is understood to be a two-sided congruence.

\begin{proposition}
  \label{prop:cong-def}
  Let $\rho$ be a congruence on a semigroup $S$.  If $(x, y), (s, t) \in \rho$,
  then $(xs, yt) \in \rho$.
  \begin{proof}
    Since $\rho$ is a left congruence, $xs ~\rho~ xt$, and since it is a right
    congruence, $xt ~\rho~ yt$.  Hence, by transitivity, $xs ~\rho~ yt$, as
    required.
  \end{proof}
\end{proposition}

Congruences have a property which allows new semigroups to be made from old
ones.  Consider the following definition of a quotient semigroup.

\begin{definition}
  \label{def:quotient}
  Let $S$ be a semigroup, and let $\rho$ be a congruence on $S$.  The
  \textbf{quotient semigroup} $S / \rho$ is the semigroup whose elements are the
  congruence classes of $\rho$, and whose operation $*$ is defined by
  $$[a]_\rho * [b]_\rho = [ab]_\rho,$$
  for $a, b \in S$.
\end{definition}

In order for quotient semigroups to be well-defined, the product of the two
classes must be regardless of which representatives are chosen for the classes
$[a]_\rho$ and $[b]_\rho$.  Hence consider arbitrary elements $a' \in [a]_\rho$
and $b' in [b]_\rho$.  We must have
$[a]_\rho * [b]_\rho = [a']_\rho * [b']_\rho$, so we must have
$[ab]_\rho = [a'b']_\rho$.  Since $a \rho a'$ and $b \rho b'$, we have
$ab \rho a'b'$ by Proposition \ref{prop:cong-def}, and so
$[ab]_\rho = [a'b']_\rho$ as required.  So a quotient semigroup is well-defined.
However, note that such a condition does not generally hold for left and right
congruences, so a quotient can only be taken by a two-sided congruence.

\begin{definition}
  \label{def:natural-homomorphism}
  Let $S$ be a semigroup, and let $\rho$ be a congruence on $S$.  The
  \textbf{natural homomorphism} $\pi_\rho$ or $\pi$ is the map from $S$ to
  $S / \rho$ which takes an element to its $\rho$-class:
  $$\pi_\rho: x \mapsto [x]_\rho.$$
\end{definition}

Congruences have long been an important area of study in semigroup theory.
Perhaps the most important feature of two-sided congruences is that they
determine the homomorphic images of a semigroup, and therefore describe an
important part of a semigroup's structure.  Consider the following theorem.

\begin{theorem}
  \label{thm:first-isomorphism}
  Let $S$ and $T$ be semigroups, and let $\phi$ be a homomorphism from $S$ to
  $T$.  Then the kernel of $\phi$ is a congruence on $S$, and the image of
  $\phi$ is isomorphic to the quotient semigroup $S / \ker{\phi}$.
  $$
  \begin{tikzcd}
    S \ar[d, two heads, "\pi"'] \ar[r, "\phi"] & T \\
    S / \ker{\phi} \ar[ur, dashed, hook, "\bar\phi"']
  \end{tikzcd}
  $$
\end{theorem}

These ideas fit closely with the concept of semigroup \textit{presentations},
which we can describe after the concept of \textit{free semigroups}.

\begin{definition}
  \label{def:free}
  Let $X$ be a set.  The \textbf{free monoid} over $X$ is denoted by $X^*$, and
  consists of all finite sequences of elements in $X$, with the operation of
  concatenation.  The \textbf{free semigroup} $X^+$ is the subsemigroup of $X^*$
  consisting of sequences of length at least $1$.
\end{definition}

When we consider free semigroups and monoids, the set $X$ is usually referred to
as an \textit{alphabet}, its elements as \textit{letters}, and sequences of
letters as \textit{words}.

We now describe a concept key to Chapter \ref{chap:pairs} as well as to
semigroup presentations, that of \textit{generating pairs}.

\begin{definition}
  \label{def:gen-pairs}
  Let $S$ be a semigroup and let $R$ be a subset of $S \times S$.
  \begin{itemize}
  \item
    The \textbf{left congruence generated by} $R$ is the least left congruence
    (with respect to containment) which contains $R$ as a subset.
  \item
    The \textbf{right congruence generated by} $R$ is the least right congruence
    (with respect to containment) which contains $R$ as a subset.
  \item
    The \textbf{congruence generated by} $R$ is the least congruence
    (with respect to containment) which contains $R$ as a subset.  It is denoted
    by $R^\sharp$.
  \end{itemize}
\end{definition}

\begin{definition}
  \label{def:presentation}
  Let $X$ be a set, and $R$ be a subset of $X^+ \times X^+$.  The
  \textbf{semigroup presentation} $\pres X R$ is a description of the quotient
  semigroup $X^+ / R^\sharp$.
\end{definition}
