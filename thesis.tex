\documentclass[a4paper]{report}

% Packages
\usepackage[nodisplayskipstretch]{setspace} % must be before algorithm & float
\usepackage[noend]{algpseudocode}
\usepackage[textformat=period]{caption}
\usepackage[hidelinks]{hyperref}
\usepackage[UKenglish]{isodate}
\usepackage[refpage, intoc]{nomencl}
\usepackage[nottoc]{tocbibind}
\usepackage{algorithm}
\usepackage{aliascnt}
\usepackage{amsfonts}
\usepackage{amsmath}
\usepackage{amssymb}
\usepackage{amsthm}
\usepackage{array}
\usepackage{cancel}
\usepackage{colortbl}
\usepackage{dot2texi}
\usepackage{draftwatermark}
\usepackage{enumerate}
\usepackage{float}
\usepackage{forest}
\usepackage{lipsum}
\usepackage{longtable}
\usepackage{makeidx}
\usepackage{mathdots}
\usepackage{mathrsfs}
\usepackage{mathtools}
\usepackage{multicol}
\usepackage{multirow}
\usepackage{pifont}
\usepackage{pstricks}
%\usepackage{showframe}
\usepackage{svg}
\usepackage{tikz-cd}
\usepackage{tikz}
\usepackage{url}
\usepackage{xcolor}
\usepackage{xfrac}

% List of notation
\renewcommand\nomname{List of notation}
\def\pagedeclaration#1{\dotfill \hyperlink{page.#1}{\nobreakspace#1}}

% Compile the index and notation
\makeindex
\makenomenclature

% Automatic bracket sizing
%\usepackage{nath}
%\delimgrowth=1
%\delimitershortfall=-1pt

% Other library imports
\usetikzlibrary{arrows,shapes}
\forestset{uf/.style={for tree={edge={<-}}}}

% Algorithmic
\algnewcommand\algorithmicbreak{\textbf{break}}
\algnewcommand\Break{\algorithmicbreak{} }

% Hyperlink references (invisible)
%\hypersetup{linkcolor=black, urlcolor=black, citecolor=black}

% Fix double spacing in arrays
\setlength{\jot}{-1ex}
\renewcommand*\arraystretch{.7}

% Watermark with git hash
\immediate\write18{git rev-parse --short HEAD > hash.info}
\immediate\write18{git diff --shortstat > changes.info}
\immediate\write18{git diff --staged --shortstat >> changes.info}
\SetWatermarkText{Commit \input{hash.info} \quad \input{changes.info}}
\SetWatermarkColor[gray]{0.75}
\SetWatermarkFontSize{0.5cm}
\SetWatermarkAngle{90}
\SetWatermarkHorCenter{3cm}

% Macros
% Text operators
\DeclareMathOperator\rank{rank}
\DeclareMathOperator\Cong{Cong}
\DeclareMathOperator\id{id}
\DeclareMathOperator\im{im}

% Green's relations
\newcommand{\HH}{\mathrel{\mathscr{H}}}
\newcommand{\LL}{\mathrel{\mathscr{L}}}
\newcommand{\RR}{\mathrel{\mathscr{R}}}
\newcommand{\DD}{\mathrel{\mathscr{D}}}
\newcommand{\JJ}{\mathrel{\mathscr{J}}}

\newcommand{\nHH}{\mathrel{\cancel{\mathscr{H}}}}
\newcommand{\nLL}{\mathrel{\cancel{\mathscr{L}}}}
\newcommand{\nRR}{\mathrel{\cancel{\mathscr{R}}}}
\newcommand{\nDD}{\mathrel{\cancel{\mathscr{D}}}}
\newcommand{\nJJ}{\mathrel{\cancel{\mathscr{J}}}}

% Special semigroups
\newcommand{\LZ}{\mathcal{LZ}}
\newcommand{\Z}{\mathcal{Z}}


% Listing figures, tables and algorithms together
\renewcommand\listfigurename{List of figures, tables and algorithms}
\makeatletter
\def\ext@algorithm{lof}
\def\ext@table{lof}
\AtBeginDocument{
  \let\l@algorithm\l@figure
  \let\c@algorithm\c@figure
  \let\thealgorithm\thefigure
  \let\l@table\l@figure
  \let\c@table\c@figure
  \let\thetable\thefigure
}
\makeatother

% Object numbering
\newtheorem{theorem}{Theorem}[chapter]
\newtheorem{lemma}[theorem]{Lemma}
\newtheorem{conjecture}[theorem]{Conjecture}
\newtheorem{proposition}[theorem]{Proposition}
\newtheorem{corollary}[theorem]{Corollary}
\theoremstyle{definition}
\newtheorem{definition}[theorem]{Definition}
\newtheorem{example}[theorem]{Example}
\newtheorem{method}[theorem]{Method}

\newcounter{common}[chapter]
\makeatletter
\let\c@algorithm\relax
\let\c@figure\relax
\let\c@table\relax
\let\c@theorem\relax
\makeatother
\newaliascnt{algorithm}{common}
\newaliascnt{figure}{common}
\newaliascnt{table}{common}
\newaliascnt{theorem}{common}
\renewcommand{\thealgorithm}{\thechapter.\arabic{algorithm}}
\renewcommand{\thefigure}{\thechapter.\arabic{figure}}
\renewcommand{\thetable}{\thechapter.\arabic{table}}
\renewcommand{\thetheorem}{\thechapter.\arabic{theorem}}

% Displaying the title
\makeatletter
\def\printtitle{{\@title}}
\makeatother

% Sequential page numbering
\let\oldsetcounter=\setcounter
\renewcommand\setcounter[2]{%
    \def\arg{#1}\def\pg{page}%
    \ifx\arg\pg\else\oldsetcounter{#1}{#2}\fi}

% % Poem in table of contents
% \usepackage{xpatch}
% \newcommand{\mytextbeforetocheading}{
%   \begingroup
%   \vskip3\baselineskip
%   \centering
%   \renewcommand{\arraystretch}{1.3}
%   \begin{tabular}{>{\phantom{mmmmmmmm}}l}
%   Before we start to write this thesis, \\
%   We'll first define some bits and pieces. \\
%   Second, we'll use parallel code,\\
%   To cut our run-times down a load.\\
%   Third, we'll look at other ways\\
%   To view a congruence nowadays.\\
%   Fourth -- if you're still reading, that is --\\
%   We'll show a way to find a lattice.\\
%   Fifth, the congruences we'll find\\
%   of the Motzkin monoid and its kind.\\
%   Last, $S$'s congruences: are there any?\\
%   And if there are, then just how many?
%   \end{tabular}
%   \endgroup
% }
% \makeatletter
% \xpatchcmd{\tableofcontents}{
%   \chapter
% }{
%   \begingroup
%    \def\@makeschapterhead##1{
%     \vspace*{50\p@}
%     {\parindent \z@ \raggedright
%       \normalfont
%       \interlinepenalty\@M
%       {\Huge \bfseries  ##1\par\nobreak}

%       \mytextbeforetocheading
%       \vskip 40\p@
%     }}
%   \chapter
% }{\typeout{success}}{\typeout{failure}}
% \xapptocmd{\tableofcontents}{\endgroup}{}{} % Close the group
% \makeatother

\title{Semigroup Congruences:\\
       Computational Techniques and Theoretical Results}
\author{Michael Torpey}

\begin{document}

\begin{titlepage}
  \centering

  \null
  \vspace{2em}

  {\Huge \textsc{\printtitle} \par}
  \vspace{7em}

  {\huge Michael Torpey}
  \vspace{6em}

  \includegraphics[height=14.8em,keepaspectratio,clip=true]{pics/arms}
  \vspace{6em}

  {\large This thesis is submitted in partial fulfilment for \\
    the degree of PhD at the University of St Andrews \par}
  \vspace{7em}

  {\Large \today}
\end{titlepage}
 \null \newpage
\noindent{\Huge \textbf{Declarations}}

\section*{Candidate's declarations}

\noindent
I, Michael Torpey, hereby certify that this thesis, which is approximately
%TODO: work out the word count
0
words in length, has been written by me, and that it is the record of work
carried out by me, or principally by myself in collaboration with others as
acknowledged, and that it has not been submitted in any previous application for
a higher degree. \\

\noindent
I was admitted as a research student and as a candidate for the degree of Doctor
of Philosophy in September 2014; the higher study for which this is a record was
carried out in the University of St Andrews between 2014 and 2018. \\

\noindent
Date:\makebox[7em]{\dotfill}
Signature of candidate:\dotfill
\\
\vspace{-0.5em}

\section*{Supervisor's declaration}

\noindent
I hereby certify that the candidate has fulfilled the conditions of the
Resolution and Regulations appropriate for the degree of Doctor of Philosophy in
the University of St Andrews and that the candidate is qualified to submit this
thesis in application for that degree. \\

\noindent
Date:\makebox[7em]{\dotfill}
Signature of supervisor:\dotfill
\\
\vspace{-0.5em}

\section*{Permission for publication}

\noindent
In submitting this thesis to the University of St Andrews I understand that I am
giving permission for it to be made available for use in accordance with the
regulations of the University Library for the time being in force, subject to
any copyright vested in the work not being affected thereby.  I also understand
that the title and the abstract will be published, and that a copy of the work
may be made and supplied to any bona fide library or research worker, that my
thesis will be electronically accessible for personal or research use unless
exempt by award of an embargo as requested below, and that the library has the
right to migrate my thesis into new electronic forms as required to ensure
continued access to the thesis. I have obtained any third-party copyright
permissions that may be required in order to allow such access and migration, or
have requested the appropriate embargo below. \\

\noindent
The following is an agreed request by candidate and supervisor regarding the
publication of this thesis:
%TODO
no embargo on print copy or electronic copy. \\

\noindent
Date:\makebox[7em]{\dotfill}
Signature of candidate:\dotfill
\\

\noindent
Date:\makebox[7em]{\dotfill}
Signature of supervisor:\dotfill
 \null \newpage

\doublespacing

\begin{abstract}

  Computational semigroup theory is an area of research that is subject to
  growing interest.  The development of semigroup algorithms allows for new
  theoretical results to be discovered, which in turn informs the creation of
  more new algorithms.  Groups have benefitted from this cycle since before the
  invention of electronic computers, and the popularity of computational group
  theory has resulted in a rich and detailed literature.  Computational
  semigroup theory is a less developed field, but recent work has resulted in a
  variety of algorithms, and some important pieces of software such as the
  \Semigroups{} package for \GAP{}.

  Congruences are an important part of semigroup theory, describing a
  semigroup's homomorphic images in the same way as a group's normal subgroups.
  There currently exist few algorithms for computing with semigroup congruences.
  However, a number of results about alternative representations for
  congruences, as well as existing algorithms that can be borrowed from group
  theory, make congruences a fertile area for improvement.  In this thesis, we
  first consider computational techniques that can be applied to the study of
  congruences, and then present some results that have been produced or
  precipitated by applying these techniques to examples.

  After some preliminary theory, we start with a new parallel approach to
  computing with congruences specified by generating pairs.  We then consider
  alternative ways of representing a congruence, using intermediate objects such
  as normal subgroups and linked triples.  We also present an algorithm for
  computing the entire congruence lattice of a finite semigroup.  In the second
  part of the thesis, we classify the congruences of the Motzkin monoid and
  other monoids of bipartitions, as well as the principal factors of the full
  transformation monoid and several related monoids.  Finally, we consider the
  number of congruences a semigroup can have, and examine the congruences on
  semigroups of size up to seven.
  
  \thispagestyle{plain}
\end{abstract}


\singlespacing
\tableofcontents
\listoffigures
\doublespacing

\chapter*{Preface}
\addcontentsline{toc}{chapter}{Preface}

\lipsum[1-15]

\chapter*{Acknowledgements}
\addcontentsline{toc}{chapter}{Acknowledgements}

Attempting to complete a PhD has been a great undertaking, and in completing
this thesis I am nearing the end of an important chapter in my life.  The years
I have spent as a postgraduate researcher have probably been the happiest of my
life, but at times the work involved has been tough, and without the support of
people around me I certainly couldn't have made it this far.  Almost everyone I
have met and got to know during this period has touched my life in a positive
way, but there are a few people in particular that I wish to thank.

Firstly, I would like to thank my supervisor James D.~Mitchell, for his honesty
and friendliness, and for the many hours he has spent correcting my work and
making me a better mathematician.  Secondly, I would like to thank my friend and
office-mate Wilf Wilson, whose wonderful company has kept me from falling asleep
through many weary afternoons of writing and coding.  I am also indebted to the
Engineering and Physical Sciences Research Council (EPSRC), whose generous grant
has allowed me to pursue computational semigroup theory freely for the last four
years.

Finally, I wish to thank Claire Young.  She has been the most important part of
my life throughout my postgraduate career, and her love and support during the
tougher moments of this PhD have given me the motivation to overcome what
sometimes felt like insurmountable obstacles.

\begin{flushright}
  Michael Torpey

  \singlespacing
  \textit{St Andrews \\July 2018}
\end{flushright}

% TODO: Further acknowledgements


\chapter{Introduction}
\label{chap:intro}

Things to define/explain:

\begin{itemize}
\item Lattices of congruences (intersection, join, etc.)
\end{itemize}

\begin{definition}
  \label{def:semigroup}
  A \textbf{semigroup} is a set $S$ together with
  a binary operation $*: S \times S \to S$ such that
  $$(x * y) * z = x * (y * z),$$
  for all $x, y, z \in S$.
\end{definition}

\begin{definition}
  \label{def:congruence}
  Let $S$ be a semigroup, and let $\rho$ be an equivalence relation on $S$.  The
  relation $\rho$ is:
  \begin{itemize}
  \item a \textbf{left congruence} if $(x, y) \in \rho$ implies that
    $(ax, ay) \in \rho$ for all $a \in S$;
  \item a \textbf{right congruence} if $(x, y) \in \rho$ implies that
    $(xa, ya) \in \rho$ for all $a \in S$;
  \item a \textbf{two-sided congruence} if it is both a left congruence and a
    right congruence.
  \end{itemize}
\end{definition}

If we talk about a \textit{congruence} without specifying that it is left or
right, it is understood to be a two-sided congruence.

\begin{proposition}
  \label{prop:cong-def}
  Let $\rho$ be a congruence on a semigroup $S$.  If $(x, y), (s, t) \in \rho$,
  then $(xs, yt) \in \rho$.
  \begin{proof}
    Since $\rho$ is a left congruence, $xs ~\rho~ xt$, and since it is a right
    congruence, $xt ~\rho~ yt$.  Hence, by transitivity, $xs ~\rho~ yt$, as
    required.
  \end{proof}
\end{proposition}

Congruences have a property which allows new semigroups to be made from old
ones.  Consider the following definition of a quotient semigroup.

\begin{definition}
  \label{def:quotient}
  Let $S$ be a semigroup, and let $\rho$ be a congruence on $S$.  The
  \textbf{quotient semigroup} $S / \rho$ is the semigroup whose elements are the
  congruence classes of $\rho$, and whose operation $*$ is defined by
  $$[a]_\rho * [b]_\rho = [ab]_\rho,$$
  for $a, b \in S$.
\end{definition}

In order for quotient semigroups to be well-defined, the product of the two
classes must be regardless of which representatives are chosen for the classes
$[a]_\rho$ and $[b]_\rho$.  Hence consider arbitrary elements $a' \in [a]_\rho$
and $b' in [b]_\rho$.  We must have
$[a]_\rho * [b]_\rho = [a']_\rho * [b']_\rho$, so we must have
$[ab]_\rho = [a'b']_\rho$.  Since $a \rho a'$ and $b \rho b'$, we have
$ab \rho a'b'$ by Proposition \ref{prop:cong-def}, and so
$[ab]_\rho = [a'b']_\rho$ as required.  So a quotient semigroup is well-defined.
However, note that such a condition does not generally hold for left and right
congruences, so a quotient can only be taken by a two-sided congruence.

\begin{definition}
  \label{def:natural-homomorphism}
  Let $S$ be a semigroup, and let $\rho$ be a congruence on $S$.  The
  \textbf{natural homomorphism} $\pi_\rho$ or $\pi$ is the map from $S$ to
  $S / \rho$ which takes an element to its $\rho$-class:
  $$\pi_\rho: x \mapsto [x]_\rho.$$
\end{definition}

Congruences have long been an important area of study in semigroup theory.
Perhaps the most important feature of two-sided congruences is that they
determine the homomorphic images of a semigroup, and therefore describe an
important part of a semigroup's structure.  Consider the following theorem.

\begin{theorem}
  \label{thm:first-isomorphism}
  Let $S$ and $T$ be semigroups, and let $\phi$ be a homomorphism from $S$ to
  $T$.  Then the kernel of $\phi$ is a congruence on $S$, and the image of
  $\phi$ is isomorphic to the quotient semigroup $S / \ker{\phi}$.
  $$
  \begin{tikzcd}
    S \ar[d, two heads, "\pi"'] \ar[r, "\phi"] & T \\
    S / \ker{\phi} \ar[ur, dashed, hook, "\bar\phi"']
  \end{tikzcd}
  $$
\end{theorem}

These ideas fit closely with the concept of semigroup \textit{presentations},
which we can describe after the concept of \textit{free semigroups}.

\begin{definition}
  \label{def:free}
  Let $X$ be a set.  The \textbf{free monoid} over $X$ is denoted by $X^*$, and
  consists of all finite sequences of elements in $X$, with the operation of
  concatenation.  The \textbf{free semigroup} $X^+$ is the subsemigroup of $X^*$
  consisting of sequences of length at least $1$.
\end{definition}

When we consider free semigroups and monoids, the set $X$ is usually referred to
as an \textit{alphabet}, its elements as \textit{letters}, and sequences of
letters as \textit{words}.

We now describe a concept key to Chapter \ref{chap:pairs} as well as to
semigroup presentations, that of \textit{generating pairs}.

\begin{definition}
  \label{def:gen-pairs}
  Let $S$ be a semigroup and let $R$ be a subset of $S \times S$.
  \begin{itemize}
  \item
    The \textbf{left congruence generated by} $R$ is the least left congruence
    (with respect to containment) which contains $R$ as a subset.
  \item
    The \textbf{right congruence generated by} $R$ is the least right congruence
    (with respect to containment) which contains $R$ as a subset.
  \item
    The \textbf{congruence generated by} $R$ is the least congruence
    (with respect to containment) which contains $R$ as a subset.  It is denoted
    by $R^\sharp$.
  \end{itemize}
\end{definition}

\begin{definition}
  \label{def:presentation}
  Let $X$ be a set, and $R$ be a subset of $X^+ \times X^+$.  The
  \textbf{semigroup presentation} $\pres X R$ is a description of the quotient
  semigroup $X^+ / R^\sharp$.
\end{definition}


\part{Computational techniques}
\label{part:algorithms}
\chapter{Parallel method for a congruence by generating pairs}
\label{chap:pairs}

A congruence is a binary relation, and is therefore formally described as a set
of pairs.  In a computational setting, it is rarely practical to keep track of
every pair in a congruence---a congruence on a semigroup of size $n$ contains
$n^2$ pairs in the worst case, and on an infinite semigroup contains an infinite
number of pairs.  A congruence can be described in more concise ways utilising
known theory: for example, taking advantage of its being an equivalence relation
and recording only its equivalence classes; or in the case of a Rees congruence,
storing a generating set for the ideal which defines it.  A variety of different
ways to describe a congruence are described in Chapter \ref{chap:converting},
along with ways to convert one to another.  However, a congruence is still just
a set of pairs, and by reducing the number of pairs we store, we can describe a
congruence very concisely using them.

A congruence $\rho$ is \textit{generated by} a set of pairs $R$ if it consists
of only the pairs in $R$ along with the pairs required by the axioms of a
congruence (reflexivity, symmetry, transitivity and compatibility).  Thus a
congruence can be described completely by storing only a few pairs.

Very few pairs indeed.  Most congs are principal.

Most generic - other systems for inverse smgps, groups, simple smgps (etc.?)

One pair implies which others? Interesting.

Left and right as well.

Parallel processing is great

Among algorithms, some are better than others, but we don't know in advance.

\section{Types of semigroups}

Concrete vs f.p.

Monoids are trivially different.

\section{Todd-Coxeter}
\label{sec:tc}

Background

The TC algorithm

Pre-filling the table

Semigroups/congs it works/works best on

Complexity

\section{KBFP}
\label{sec:kbfp}

Background

The KBFP algorithm

Semigroups/congs it works/works best on

Complexity

\section{Pair orbit enumeration}
\label{sec:p}

Background

The P algorithm

Using Knuth-Bendix: KBP

Semigroups/congs it works/works best on

Complexity

\section{Running in parallel}

How do we tie together all the different algorithms?

\section{Implementation}

Practical considerations in libsemigroups

Showing off speed

Drawbacks

\chapter{Converting between different congruence representations}
\label{chap:converting}

A congruence is a binary relation, and therefore is formally described as a set
of pairs---a subset of $S \times S$..  In a computational setting, it would be
possible to store a congruence by simply storing every one of these pairs, so
long as $S$ was finite.  However, this could well use a lot of storage---even
the trivial congruence would use $O(|S|)$ space, and in general a congruence
could use $O(|S|^2)$ space.

A congruence is a binary relation, and therefore is formally described as a set
of pairs.  In a computational setting, it is rarely practical to keep track of
every pair in a congruence; a congruence on a semigroup of size $n$ contains
$n^2$ pairs in the worst case, and on an infinite semigroup contains an infinite
number of pairs.  A congruence can be described in more concise ways: for
example, taking advantage of it being an equivalence relation and recording only
its equivalence classes; or in the case of a Rees congruence, storing a
generating set for the ideal which defines it.  A variety of different ways to
describe a congruence are explained in Chapter \ref{chap:converting}, along with
ways to convert from one to another.  However, a congruence is still just a set
of pairs, and by reducing the number of pairs we store, we can often describe a
congruence very concisely using them.

In Chapter \ref{chap:pairs} we looked at how a semigroup congruence can be
represented by a set of generating pairs.  In fact,

\section{Ways of representing a congruence}
\subsection{Generating pairs}
\subsection{Completely (0-)simple: linked triples}
\subsection{Inverse semigroups: kernel-trace pairs}
\subsection{Rees congruences}
A \textbf{Rees congruence} is a congruence on a semigroup $S$ with a distinguished
congruence class $I$ which is a two-sided ideal of $S$, and in which every other
congruence class is a singleton.  We may write this congruence as $\rho_I$, and
we may write its quotient $S/\rho_I$ as $S/I$.

Some or all of a semigroup's congruences may be Rees: in particular, since $S$
is an ideal of $S$, the universal congruence $S \times S$ is a Rees congruence.

Another type of congruence is one given by a set of generating pairs: if $R
\subset S \times S$, then we let $R^\sharp$ be the least congruence on $S$
containing all the pairs in $R$.  We describe $R$ as a set of \textbf{generating
  pairs} for $R^\sharp$, and any congruence on $S$ can be described in this way.

\subsection{Regular semigroups: something like kernel-trace?}

\section{Generating pairs of a Rees congruence}
\label{sec:rees-to-pairs}
A natural question, given an ideal $I$, is how to find a set of generating pairs
for $\rho_I$.

\begin{theorem}
  Let $S$ be a semigroup, $I$ an ideal of $S$, and $X$ an ideal generating set
  for $I$.  Also let $M$ be the minimal ideal of $S$ (which may or may not be
  equal to $I$).  Then $$X \times M$$ is a set of generating pairs for the Rees
  congruence $\rho_I$.
  \begin{proof}
    Let $\rho$ be the congruence generated by $X \times M$.  First we show that
    $\rho \subseteq \rho_I$, and then that $\rho_I \subseteq \rho$.

    Let $(i,m) \in X \times M$.  We have $X \subseteq I$ since $X$ is a
    generating set for $I$, and $M \subseteq I$ since $M$ is contained in any
    ideal of $S$.  Hence $i$ and $m$ both lie in $I$, so they are in the same
    class of the Rees congruence: $(i,m) \in \rho_I$.  Hence $X \times M
    \subseteq \rho_I$, and so $\rho$ (the least congruence containing $X \times
    M$) must also be contained in $\rho_I$.  Hence $\rho \subseteq \rho_I$.

    Now let $(a,b) \in \rho_I$; we wish to show that $(a,b) \in \rho$.  If $a=b$
    then we certainly have $(a,b) \in \rho$.  Otherwise we must have $a,b \in
    I$.  Since $X$ \textit{generates} $I$, we have $I = S^1XS^1$.  Therefore we
    can write
    $$a = s_1x_1t_1, \quad b = s_2x_2t_2,$$
    for some $x_1,x_2 \in X$ and $s_1,s_2,t_1,t_2 \in S^1$.

    Now choose some $m \in M$.  By definition $(x_1,m), (x_2,m) \in \rho$ since
    $X \times M \subseteq \rho$, and
    by the compatibility properties of a congruence,
    $$(s_1x_1t_1,s_1mt_1), (s_2x_2t_2,s_2mt_2) \in \rho.$$

    Since $m \in M$, we must have $s_1mt_1,s_2mt_2 \in M$.  Let $x_0$ be an
    arbitrary element of $X$.
    We see $(x_0,s_1mt_1), (x_0,s_2mt_2) \in X \times M$, and so by transitivity
    $(s_1mt_1, s_2mt_2) \in \rho$.
    Hence
    $$a ~=~ s_1x_1t_1 ~\rho~ s_1mt_1 ~\rho~ s_2mt_2 ~\rho~ s_2x_2t_2 = b,$$
    and $(a,b) \in \rho$ as required.
  \end{proof}
\end{theorem}

\section{Kernel and trace from generating pairs}
\label{sec:pairs-to-kertr}

\section{Generating pairs from kernel and trace}
\label{sec:kertr-to-pairs}

\chapter{Finding a semigroup's congruence lattice}
\label{chap:lattice}

We can learn a lot about a semigroup's structure by examining its congruences:
they describe a semigroup's homomorphic images, and quotient semigroups, as
explained in Section \ref{sec:intro-congs}.  For this reason, it is of great
interest to be able to produce a complete list of congruences on a given
semigroup.

In group theory, we study normal subgroups instead of studying congruences
directly (see Section \ref{sec:normal-subgroups}).  Several algorithms exist for
computing a given group's normal subgroups, and therefore its congruences---see,
for example, \cite{hulpke_1998} for one of the methods used in GAP.  However,
since in the general case a semigroup's congruences are much less regular than
in the group case (for example, congruence classes may have different sizes) it
is certainly more computationally difficult to compute a complete list of
congruences for a given semigroup.

In this chapter, we present a method for calculating all the congruences of a
finite semigroup.  This algorithm takes advantage of the fact that congruences
lie in a lattice with respect to containment ($\subseteq$), intersection
($\cap$) and join ($\vee$).  It computes the lattice structure while it computes
the congruences themselves, and so the lattice structure is returned as an
output of the algorithm, along with the set of congruences.  This algorithm was
used as a starting point for the work described in Chapter \ref{chap:motzkin}.

In Section \ref{sec:lattice-algorithm} we give the algorithm in pseudo-code, and
explain how it works.  In Section \ref{sec:lattice-implementation} we outline
some practical concerns for implementing the algorithm, with particular
reference to how it is implemented in the Semigroups package \cite{semigroups}
for GAP \cite{gap}.  And finally, in Section \ref{sec:lattice-examples}, we
present some examples of lattices which have been computed using this algorithm.

\section{The algorithm}
\label{sec:lattice-algorithm}

For the purposes of this section, we will make the following definition.

\begin{definition}
  \label{def:congruence-poset}
  \index{congruence!poset}
  A \textbf{congruence poset} on a semigroup $S$ is a pair $(\Gamma, \PO)$
  where:
  \begin{itemize}
  \item $\Gamma$ is a set of congruences on $S$; and
  \item $\PO$ is $\subseteq$, the partial order of containment on $\Gamma$.
  \end{itemize}
\end{definition}

Recall that a partial order is defined as a relation that is reflexive
($x \leq x$), anti-symmetric ($x \leq y$ and $y \leq x$ if and only if $x = y$),
and transitive ($x \leq y$ and $y \leq z$ implies $x \leq z$).
\index{partial order}
Hence $\PO$ will be a set of pairs of the form $(\rho, \sigma)$, where $\rho$
and $\sigma$ are both congruences on $S$, and $\rho \subseteq \sigma$.  If
$\Gamma$ is the set of all congruences on $S$, then $(\Gamma, \PO)$ will be a
lattice by Theorem \ref{thm:congruence-lattice}, and two congruences $\rho$ and
$\sigma$ will have an intersection $\rho \cap \sigma$ and a join
$\rho \vee \sigma$ in $\Gamma$.

We first present an algorithm to calculate the principal congruences of a
semigroup, along with their partial ordering $\subseteq$.  This is a congruence
poset, but since it may not contain all the congruences on the given semigroup,
it may not be a lattice.  We call the algorithm \textsc{PrincCongPoset}, and we
give pseudo-code for it in Algorithm \ref{alg:princ-cong-poset}.

\begin{algorithm}
  \caption{The \textsc{PrincCongPoset} algorithm}
  \label{alg:princ-cong-poset}
  \index{PrincCongPoset@\textsc{PrincCongPoset}}
  \begin{algorithmic}[1]
    \Require $S$ a finite semigroup
    \Procedure{PrincCongPoset}{$S$}
      \State $\Gamma := \varnothing$
      \Comment{Set of congruences}
      \State $\PO := \varnothing$
      \Comment{Partial order ($\subseteq$) on congruences}
      \For{$(x,y) \in S \times S$}
        \State $P := \left\{\big((x,y)^\sharp, (x,y)^\sharp\big)\right\}$
        \Comment{Possible new pairs for $\PO$}
        \For{$(a,b)^\sharp \in \Gamma$}
          \If{$(x,y) \in (a,b)^\sharp$}
            \If{$(a,b) \in (x,y)^\sharp$}
              \State Skip to the next pair $(x,y)$
              \Comment{$(a,b)^\sharp = (x,y)^\sharp$}
            \Else
              \State $P \gets P \cup
                \left\{\big((x,y)^\sharp, (a,b)^\sharp\big)\right\}$
            \EndIf
          \ElsIf{$(a,b) \in (x,y)^\sharp$}
              \State $P \gets P \cup
                \left\{\big((a,b)^\sharp, (x,y)^\sharp\big)\right\}$
          \EndIf
        \EndFor
        \State $\Gamma \gets \Gamma \cup \{(x,y)^\sharp\}$
        \State $\PO \gets \PO \cup P$
        % \State Add $(x,y)^\sharp \leq (a,b)^\sharp$ for each $(a,b)^\sharp$ in $P$
        % \State Add $(a,b)^\sharp \leq (x,y)^\sharp$ for each $(a,b)^\sharp$ in $C$
        % \For{$(a,b)^\sharp \in P$}
        %   \State Set $(x,y)^\sharp \leq (a,b)^\sharp$
        % \EndFor
        % \For{$(a,b)^\sharp \in C$}
        %   \State Set $(a,b)^\sharp \leq (x,y)^\sharp$
        % \EndFor
        % \State $\PO \gets \PO \cup
        % \big\{\big((x,y)^\sharp, (a,b)^\sharp\big) : (a,b)^\sharp \in P\big\}$
        % \State $\PO \gets \PO \cup
        % \big\{\big((a,b)^\sharp, (x,y)^\sharp\big) : (a,b)^\sharp \in C\big\}$
      \EndFor
      \State \Return $(\Gamma, \PO)$
    \EndProcedure
  \end{algorithmic}
\end{algorithm}

The idea of \textsc{PrincCongPoset} is to go through each pair in $S \times S$,
and consider the congruence generated by that pair.  We compare each new
congruence to the congruences we have found so far, establishing which of them
it contains, and which it is contained in.  If it turns out to be equal to a
congruence already found, we drop it immediately; if it turns out to be an
entirely new congruence, then we can add it to the list $\Gamma$ of congruences,
and add pairs to $\PO$ that describe how it compares to the other congruences.
Since each new congruence is compared to every previously found congruence,
every possible appropriate pair is added to $\PO$, and we are therefore
guaranteed that $\PO$ will be equal to the containment relation ($\subseteq$) by
the end of the algorithm.

One positive outcome of using generating pairs in this way is that we can use
the result
$$(a,b)^\sharp \subseteq (x,y)^\sharp \quad\iff\quad (a,b) \in (x,y)^\sharp$$
for any two pairs $(a,b), (x,y) \in S \times S$.  Hence, in order to compare the
two congruences comprehensively, we only need to test the presence of one pair
in each congruence: $(a,b) \in (x,y)^\sharp$ and $(x,y) \in (a,b)^\sharp$.
Testing the presence of a given pair in a congruence is likely to be faster
than, for example, exhaustively computing its congruence classes.  A general
algorithm for testing whether a given pair lies in a congruence specified by
generating pairs is described in Chapter \ref{chap:pairs}; in some cases this
can be improved by first converting the congruence to another representation, as
described in Chapter \ref{chap:converting}.

Our second algorithm is called \textsc{JoinClosure}.  This algorithm takes a
congruence poset $(\Gamma, \PO)$ as its argument, and returns the congruence
poset containing all the congruences in $\Gamma$ along with all their joins.
That is, for any collection of $k$ congruences
$(\rho_i)_{1 \leq i \leq k}$ from $\Gamma$, the output of
\textsc{JoinClosure} will contain the congruence
$$\bigvee_{1 \leq i \leq k} \rho_i
\quad=\quad \rho_1 \vee \rho_2 \vee \ldots \vee \rho_k.$$
Pseudo-code for \textsc{JoinClosure} is shown in Algorithm
\ref{alg:join-closure}.

\begin{algorithm}
  \caption{The \textsc{JoinClosure} algorithm}
  \label{alg:join-closure}
  \index{JoinClosure@\textsc{JoinClosure}}
  \begin{algorithmic}[1]
    \Require
    $S$ a finite semigroup,
    $(\Gamma,\PO)$ a congruence poset on $S$,
    all congruences in $\Gamma$ have generating pairs
    
    \Procedure{JoinClosure}{$(\Gamma,\PO)$}
      \State $\Gamma_I := \Gamma$ \Comment{Initial congruences}
      \State $\Gamma_N := \Gamma$ \Comment{New congruences (to be joined)}
      \While{$\Gamma_N \neq \varnothing$}
        \For{$\rho_N \in \Gamma_N$}
          \State $\Gamma_N \gets \Gamma_N \setminus \{\rho_N\}$
          \For{$\rho_I \in \Gamma_I$}
            \State $\rho := \rho_N \vee \rho_I$
            \State $P := \{(\rho, \rho)\}$
            \Comment{Possible new pairs for $\PO$}
            \For{$\sigma \in \Gamma$}
              \If{$\rho \subseteq \sigma$}
                \If{$\sigma \subseteq \rho$}
                  \State Skip to the next $\rho_I$
                  \Comment{$\rho = \sigma$}
                \Else
                  \State $P \gets P \cup \{(\rho, \sigma)\}$
                \EndIf
              \ElsIf{$\sigma \subseteq \rho$}
                \State $P \gets P \cup \{(\sigma, \rho)\}$
              \EndIf
            \EndFor
            \State $\Gamma \gets \Gamma \cup \{\rho\}$
            \State $\Gamma_N \gets \Gamma_N \cup \{\rho\}$
            \State $\PO \gets \PO \cup P$
          \EndFor
        \EndFor
      \EndWhile
      \State \Return $(\Gamma, \PO)$
    \EndProcedure
  \end{algorithmic}
\end{algorithm}

The \textsc{JoinClosure} algorithm works by keeping a list $\Gamma_I$ of
``initial'' congruences (the ones we started with) and another list $\Gamma_N$
of congruences needing to be joined.  To start with, these are both equal to the
input $\Gamma$.  Each congruence in $\Gamma_N$ in turn is joined with each
initial congruence from $\Gamma_I$, and we check whether this join $\rho$ is a
new congruence or equal to one we have already found.  This check is done in a
similar way to \textsc{PrincCongPoset}, by checking $\rho \subseteq \sigma$ as
well as $\sigma \subseteq \rho$: if both are true, then we conclude that
$\rho = \sigma$ and therefore we don't have a new congruence; if just one is
true, we record the appropriate pair, and if $\rho$ is confirmed as a new
congruence, we add it to $\PO$ later.

In this algorithm, unlike in \textsc{PrincCongPoset}, we may encounter
congruences with more than one generating pair.  Hence, for two congruences
$\rho$ and $\sigma$, we cannot find out whether $\rho \subseteq \sigma$ in quite
the same way as we did in that algorithm.  We have one useful result: if
$\mathbf{R}$ and $\mathbf{S}$ are sets of generating pairs, then
$$\Rs \subseteq \mathbf{S}^\sharp \quad\iff\quad
\R \subseteq \mathbf{S}^\sharp,$$
so we only have to check containment of generating pairs in order to check
containment of congruences.  However, a congruence may have many generating
pairs, so in some cases this check may take a long time.  For this reason, if
there is an alternative way of representing the congruences (for example,
another representation from Chapter \ref{chap:converting}) then it may be
quicker to use a containment ($\subseteq$) method specific to that
representation.  For example, if $S$ is a 0-simple semigroup, then our two
congruences will have linked triples $(N_1,\sS_1,\tT_1)$ and $(N_2,\sS_2,\tT_2)$
respectively; instead of checking containment of generating pairs, we can check
whether $N_1 \leq N_2$, $\sS_1 \subseteq \sS_2$ and $\tT_1 \subseteq \tT_2$.

Each time we find a new congruence, we add it to $\Gamma_N$, and then later we
take its join with each initial congruence.  Hence, any congruence which can be
built up as the join of congruences in the initial list is eventually found.
For example, if there exists some congruence $\tau$ equal to
$\rho_1 \vee \rho_2 \vee \rho_3$, with $\rho_1,\rho_2,\rho_3 \in \Gamma_I$, then
we will find the congruence $\rho_1 \vee \rho_2$ on the first run of the while
loop, and it will be added to $\Gamma_N$.  Then, on the second run of the while
loop, we will compute $(\rho_1 \vee \rho_2) \vee \rho_3$, and hence we will find
$\tau$.  In this way, it is guaranteed that any join of congruences from
$\Gamma$ will appear in the output of
\textsc{JoinClosure}$((\Gamma, \PO))$

Now that we have described the two algorithms, it is easy to see how we can use
them to find the whole congruence lattice of a finite semigroup $S$.
\textsc{PrincCongPoset} finds all the principal congruences of $S$, and
\textsc{JoinClosure} finds all the joins of a set of congruences.  Since, in a
finite semigroup, any congruence is the join of a finite number of principal
congruences, we can produce the congruence lattice of $S$ by simply calling
$$\textsc{JoinClosure\big(PrincCongPoset($S$)\big)}.$$  This is the basis of the
function \texttt{LatticeOfCongruences} in the Semigroups package for GAP
\cite{semigroups}.

\section{Improvements}
\label{sec:lattice-improvements}

The \textsc{PrincCongPoset} algorithm as shown in Algorithm
\ref{alg:princ-cong-poset} is fairly simple, but can be modified in a few ways
to improve its performance.  Firstly, we should consider the source of
generating pairs: we iterate through all pairs $(x,y) \in S \times S$.  There
are ways in which this process is guaranteed to encounter a given congruence
twice, and therefore waste time.  For example, if we consider a pair $(x,y)$,
there is no need later to consider $(y,x)$, since it will generate the same
congruence.  Similarly there is no need to consider every reflexive pair
$(x,x)$, since each one is guaranteed to generate the trivial congruence.  Thus,
if $S$ has $n$ elements, we need only consider $\frac{1}{2}n(n-1)$ pairs, rather
than all $n^2$ pairs from $S \times S$.

Note that we could also replace $S$ here with some subset $X \subset S$, if we
wish to see what congruences can be generated only with pairs from $X \times X$.
For instance, we might be interested in congruences generated by pairs from some
ideal of $S$, and how they affect elements outside the ideal.  These questions
can be answered with minimal changes to the algorithm.

A possible improvement in both algorithms would be to use pairs already in
$\PO$, along with the axiom of transitivity, to skip certain comparisons.  For
example, if our new pair $(x,y)$ is found to be a subset of $(a,b)$, but $(a,b)$
is itself already known to be a subset of some congruence $(c,d)$, then we can
immediately add the pair $\big((x,y), (c,d)\big)$ to $P$ and we can skip the
comparison of $(x,y)$ to $(c,d)$ later in the algorithm.

Joining to known children/parents

\section{Implementation}
\label{sec:lattice-implementation}
\begin{itemize}
\item Children and Parents lists
\item Method selection for $\subseteq$
\end{itemize}

\section{Examples}
\label{sec:lattice-examples}


\part{Theoretical results}
\label{part:results}
\chapter{Congruences of the Motzkin monoid}
\label{chap:motzkin}

In Chapter \ref{chap:lattice} we explained a relatively quick way of computing
all of a semigroup's congruences, along with information about how they fit into
their lattice structure.  This was implemented in the Semigroups package
\cite{semigroups}, greatly increasing the size and complexity of semigroups
whose congruence lattices can be found using a computer.

One of the first semigroups towards which this new methodology was directed was
the bipartition monoid $\Prt_n$, whose congruence lattice was not previously
known.  Computing this lattice for the first few values of $n$ showed a lattice
with a relatively simple structure, which did not appear to increase much in
complexity as $n$ grew higher than $3$.  The congruence lattices of various
submonoids of $\Prt_n$ were also computed, and appeared to have a similar
structure.  With the rapidly increasing size of $\Prt_n$ (see Table
\ref{tab:pn-size}) it proved impractical to na\"ively calculate the congruence
lattices beyond $n=4$, but careful study of the lattices for small values of
$n$, along with those lattices computed for various submonoids of $\Prt_n$,
yielded a general classification of the congruence lattice of $\Prt_n$ for
arbitrary $n$, along with a classification of the congruence lattices of various
important submonoids.  This classification is explained and proven in
\cite{ourpaper}.

In this chapter, we will examine the structure of these congruence lattices,
focusing in particular on the Motzkin monoid $\Mot_n$, which was the present
author's particular focus when contributing to \cite{ourpaper}.  We will start
with the definition of the Motzkin monoid, then describe some preliminary
theory, then describe the Motzkin monoid's lattice of congruences, and
finally give a brief description of how these ideas can be extended to $\Prt_n$
and its other submonoids.

\section{The Motzkin monoid $\Mot_n$}
\label{sec:motzkin-monoid}
In order to define the Motzkin monoid, we must first define a \textit{planar}
bipartition.

\begin{definition}
  \label{def:planar}
  \index{planar bipartition}
  A bipartition is called \textbf{planar} if it can be represented in diagram
  form with all edges contained inside the rectangle formed by the vertices, and
  without any edges crossing.
\end{definition}

\begin{example}
  Let $\alpha = \bipart{c|c|c|c}{2-4}{1,2 &3 &4 &5}{2,5 &1 &\mc2{c}{3,4}}$ and
  $\beta = \bipart{c|c|c|c}{3-4}{2 &5 &1,3 &4}{1 &3,4 &2 &5}$.  As can be seen
  in Figure \ref{fig:planar}, $\alpha$ is planar.  However, $\beta$ cannot be
  drawn inside the rectangle without the upper block $\{1,3\}$ crossing lines
  with the transversal $\{2, 1'\}$---hence, $\beta$ is not planar.
\end{example}

\begin{figure}[h]
  \centering
  $$\alpha = \bipartdiag{\tc12\tv22\bC25 \bc34} \qquad
  \beta = \bipartdiag{\tc13 \tv21 \tv54\bc34}$$
  \caption{A planar and a non-planar bipartition}
  \label{fig:planar}
\end{figure}

We can now define the Motzkin monoid.

\begin{definition}
  \label{def:motzkin}
  \index{Motzkin monoid}
  \nomenclature[Mn]{$\Mot_n$}{Motzkin monoid}
  The \textbf{Motzkin monoid} $\Mot_n$ is the submonoid of $\Prt_n$ consisting
  of all planar bipartitions of degree $n$ in which every block has size $1$ or
  $2$.
\end{definition}

To see that this is indeed a monoid, we should observe that it is closed.  It is
easy to see that the product of two planar bipartitions is also planar, since a
double diagram as in Figure \ref{fig:bipartition-example} would contain no
crossing lines, and therefore would resolve to a product with no crossing lines.
It is also easy to see that if two bipartitions have no block larger than $2$,
their product also has no block larger than $2$: any transversal can only
contain one point in $\bn$ and one point in $\bn'$, so any transversal in the
product can only contain two points; the upper and lower blocks of the product
are inherited from the original bipartitions, so they will not break the
condition either.

The Motzkin monoid $\Mot_n$ grows much slower than its parent $\Prt_n$, having
only $\sum_{k=0}^n \binom{2n}{2k}C_k$ elements \cite[A026945]{oeis}, where $C_k$
is the $k$th Catalan number.  Its size in comparison with $\Prt_n$ is
shown in Table \ref{tab:mn-size}.

\begin{table}[h]
  \centering
  \renewcommand\arraystretch{1.0}
  \begin{tabular}{| r | r | r |}
    \hline
    $n$ & $|\Mot_n|$ & $|\Prt_n|$ \\
    \hline
     1 &           2 &                  2 \\
     2 &           9 &                 15 \\
     3 &          51 &                203 \\
     4 &         323 &              4 140 \\
     5 &       2 188 &            115 975 \\
     6 &      15 511 &          4 213 597 \\
     7 &     113 634 &        190 899 322 \\
     8 &     853 467 &     10 480 142 147 \\
     9 &   6 536 382 &    682 076 806 159 \\
    10 &  50 852 019 & 51 724 158 235 372 \\
    \hline
  \end{tabular}
  \renewcommand\arraystretch{0.7}
  \caption{Sizes of $\Mot_n$ and $\Prt_n$ for small values of $n$}
  \label{tab:mn-size}
\end{table}

The Motzkin monoid $\Mot_n$ shares a number of features with $\Prt_n$---indeed,
we will see later that its congruence lattice is very similar.  Like $\Prt_n$,
$\Mot_n$ is regular One important
similarity is in its Green's relations.  Consider the following proposition,
akin to Proposition \ref{prop:bipartition-greens}.

\begin{proposition}
  \label{prop:mn-greens}
  Let $\alpha$ and $\beta$ be bipartitions in $\Mot_n$.  The following hold:
  \begin{enumerate}[\rm(i)]
  \item $\alpha \RR \beta$ if and only if $\dom \alpha = \dom \beta$ and
    $\ker \alpha = \ker \beta$;
  \item $\alpha \LL \beta$ if and only if $\codom \alpha = \codom \beta$ and
    $\coker \alpha = \coker \beta$;
  \item $\alpha \JJ \beta$ if and only if $\rank \alpha = \rank \beta$;
  \item $J_\alpha \leq J_\beta$ if and only if $\rank \alpha \leq \rank \beta$;
  \item the ideals of $\Mot_n$ are precisely the sets
    $I_r=\{\alpha \in \Mot_n : \rank \alpha \leq r\}$ for
    $r \in \{0, \ldots, n\}$.
  \end{enumerate}
  \begin{proof}
    For (i) to (iii), see \cite[Theorem 2.4]{deg_motzkin}.  For (iv) and (v),
    see \cite[Proposition 2.6]{deg_motzkin}.
  \end{proof}
\end{proposition}

This description of the Motzkin monoid's Green's relations, and its containment
of $\JJ$-classes and ideals, will help us greatly later on.  However, one
consequence of (i) and (ii) gives $\Mot_n$ a feature which $\Prt_n$ does not
share, namely the following corollary.

\begin{corollary}
  \label{cor:mn-h-trivial}
  The Motzkin monoid $\Mot_n$ is $\HH$-trivial.
  \begin{proof}
    Let $\alpha, \beta \in \Mot_n$ such that $\alpha \HH \beta$.  This tells us
    that $\alpha \LL \beta$ and $\alpha \RR \beta$, so by Proposition
    \ref{prop:mn-greens} parts (i) and (ii), we know that $\alpha$ and $\beta$
    share the same domain, kernel, codomain and cokernel.  The upper blocks and
    lower blocks of $\alpha$ and $\beta$ must certainly be the same, since they
    are just the blocks of the kernel and cokernel that do not lie in the domain
    or codomain.  The only choice is in the transversals: which blocks in the
    domain connect to which blocks in the codomain.  In $\Prt_n$ there are
    $(\rank \alpha)!$ ways of choosing this match-up; but in $\Mot_n$ there is
    only one way possible, since we cannot allow any lines in the diagram to
    cross over.  Hence $\alpha = \beta$.
  \end{proof}
\end{corollary}


\section{Preliminaries}
\label{sec:motzkin-prelim}
We will start by defining some concepts which allow us to find certain
congruences in any semigroup: \textit{retractable ideals} (Definition
\ref{def:retractable-ideal}) and \textit{lifting congruences} (Definition
\ref{def:lifting-congruence}).  It will turn out that all the congruences of
$\Mot_n$ can be built using these two building blocks.

\begin{definition}
  \label{def:retractable-ideal}
  \index{retractable ideal} \index{retraction}
  Let $S$ be a semigroup with a minimal ideal $M$.  An ideal $I$ of $S$ is
  \textbf{retractable} if there exists a homomorphism $f: I \to M$ such that
  $xf = x$ for all $x \in M$; such a homomorphism is called a
  \textbf{retraction}.
\end{definition}

\begin{definition}
  \label{def:lifting-congruence}
  \index{lifting congruence}
  \nomenclature{$\xi$}{Lifting congruence}
  Let $S$ be a semigroup with a minimal ideal $M$.  A congruence $\xi$ on $M$ is
  a \textbf{lifting congruence} if any, and hence all, of the following
  equivalent conditions are satisfied:
  \begin{enumerate}[\rm(i)]
  \item $\Delta_S \cup \xi$ is a congruence on $S$;
  \item there exists a congruence $\zeta$ on $S$ such that
    $\xi=\zeta\cap(M\times M)$;
  \item for all $(x,y) \in \xi$ and $s \in S$, $(xs,ys),(sx,sy) \in \xi$.
  \end{enumerate}
\end{definition}

In order to use these building blocks to produce new congruences, we first need
to establish some results about them.  Note that, since $\Mot_n$ is finite, it
always has a minimal ideal.  More specifically, the minimal ideal of $\Mot_n$ is
given by $I_0 = \{\alpha \in \Mot_n : \rank \alpha = 0\}$ (see Proposition
\ref{prop:mn-greens}).  The following lemma will be used at various times
throughout this chapter.

\begin{lemma}
  \label{lem:retract-aux}
  Let $S$ be a semigroup with a regular minimal ideal $M$, and let $I$ be an
  ideal of $S$. If $f: I\rightarrow M$ is a retraction, then $(sxt)f=s(xf)t$ for
  all $x\in I$ and $s,t\in S^1$.
  \begin{proof}
    We will prove the lemma by showing that $(sx)f=s(xf)$ for any $s\in S ^ 1$
    and $x\in I$.  The equality $(xt)f=(xf)t$ is dual, and then
    $(sxt)f=s(xt)f=s(xf)t$.  Let $e\in M$ be any right identity for $xf$; such
    an~$e$ exists because $M$ is regular. Then, since $f$ is a retraction and
    $e,xe\in M$, we have $xf=(xf)e=(xf)(ef)=(xe)f=xe$.  Next, let $e_1\in M$ be
    a left identity for $(sx)f$.  Then
    $(sx)fe=e_1(sx)fe=(e_1f)(sx)f(ef)=(e_1sxe)f
     =(e_1s)f(xf)(ef)=(e_1s)f(xf)=(e_1sx)f=(e_1f)(sx)f=e_1(sx)f=(sx)f$.
    So $e$ is a right identity for $(sx)f$ as well, and hence $(sx)f=sxe=s(xf)$,
    as required.
  \end{proof}
\end{lemma}

This gives rise to an important result which we can use later when we combine
retractable ideals with lifting congruences.  Note first that, since $\Mot_n$ is
a regular semigroup, its minimal ideal is also regular.

\begin{corollary}
  \label{cor:retract-unique}
  Let $S$ be a semigroup with a regular minimal ideal $M$.  If $I$ is a
  retractable ideal of $S$, then there exists a unique retraction from $I$ to
  $M$.
  \begin{proof}
    Suppose $f,g:I\to M$ are retractions, and let $x\in I$.  Let $e\in M$ be a
    left identity for $xf$.  Using Lemma \ref{lem:retract-aux}, we see that
    $xf = e (xf) = (ex)f = ex = (ex)g = e (xg)$.  A dual argument shows that
    $xg=(xf) e'$ for some $e'$.  But then $xf = e (xg) = e (xf) e' = (xf)e'=xg$.
  \end{proof}
\end{corollary}

The effect of Corollary \ref{cor:retract-unique} is that, for a semigroup with a
regular minimal ideal, we can talk about \textit{the} retraction of a
retractable ideal without any loss of generality.  This result is the last thing
we need to use our two building blocks to produce a new congruence: a
\textit{lifted congruence}.

\begin{definition}
  \label{def:lifted-congruence}
  \index{lifted congruence}
  Let $S$ be a semigroup with a minimal ideal $M$, let $I$ be a retractable
  ideal of $S$, and let $\xi$ be a lifting congruence on $M$.  We associate to
  the pair $(I,\xi)$ the relation
  $$\zeta_{I,\xi}= \Delta_S \cup \{(x,y) \in I \times I : (xf,yf) \in \xi\},$$
  where $f:I \to M$ is the unique retraction of $I$.
  We call $\zeta_{I,\xi}$ the \textbf{lifted congruence} of $(I,\xi)$.
\end{definition}

In order to justify the name \textit{lifted congruence}, we require the
following proposition.

\begin{proposition}\label{prop:lift}
  The relation $\zeta_{I,\xi}$ in Definition \ref{def:lifted-congruence} is a
  congruence on $S$.
  \begin{proof}
    Write $\zeta=\zeta_{I,\xi}$ for brevity, and let $f:I\to M$ be the
    retraction.  Let $(x,y)\in\zeta$ and $s\in S$ be arbitrary.  We must show
    that $(xs,ys),(sx,sy)\in\zeta$.  This is clear if $x=y$, so suppose
    $x,y\in I$ and $xf\mathrel\xi yf$.  Since~$I$ is an ideal, we have
    $xs,ys\in I$.  By Lemma \ref{lem:retract-aux}, and condition (iii) of
    Definition \ref{def:lifting-congruence}, we have
    $(xs)f = (xf)s \mathrel\xi (yf)s = (ys)f$, showing that $(xs,ys)\in\zeta$.
    A dual argument shows that $(sx,sy)\in\zeta$.
\end{proof}
\end{proposition}

This construction now gives us a usable source of congruences.  All that is
required is to find lifting congruences and retractable ideals of a semigroup,
and a number of new congruences can be described.  It turns out that this is an
excellent source of congruences for $\Mot_n$, yielding every congruence on the
semigroup, as we will see later.

\section{Congruence lattice of $\Mot_n$}
\label{sec:motzkin-congs}


\section{Other monoids}
\label{sec:motzkin-other}

\chapter{Other results}
\label{chap:other}

In Chapter \ref{chap:motzkin} we classified the congruences of the Motzkin
monoid and several related diagram monoids.  This classification was achieved by
first calculating the congruence lattices for small values of $n$ using the
computational techniques described in Chapters \ref{chap:pairs} and
\ref{chap:lattice}, and then building up theory in order to prove a
classification for general $n$.  In this chapter we present some more results
about congruences that were obtained in a similar way, by first looking for
patterns in computational results, and then extending the results and attempting
to prove them for larger semigroups.  The \texttt{libsemigroups} library and the
Semigroups and \texttt{smallsemi} packages for GAP were used to carry out the
initial computations \cite{libsemigroups, semigroups, smallsemi, gap}.

\section{Congruences of principal factors}
\label{sec:princfact}

In this section, we will consider an interesting decomposition of a semigroup
related to its $\jJ$-classes $\JJ$-classes: a semigroup's \textit{principal
  factors}.  After defining this construction, we will consider the principal
factors of the full transformation monoid $\T_n$, and classify their
congruences.
After this, we will look at the principal factors of some other, somewhat
similar monoids, and classify their congruences using similar principles.

\subsection{Principal factors}
\label{sec:princfact-def}

Recall that a semigroup's $\JJ$-classes have a natural partial order $\leq$,
defined by the rule that $J_a \leq J_b$ if and only if
$S^1 J_a S^1 \subseteq S^1 J_b S^1$.  For finite semigroups we have $\jJ=\dD$,
and this partial order is shown on eggbox diagrams by the placement of
$\DD$-classes above and below each other, as in Figure \ref{fig:eggbox-diagram}.
Given a $\JJ$-class $J$ of a semigroup $S$, we can define the ideal generated by
$J$, which is given by $S^1 J S^1$.  If $J$ is not minimal, we can also define
the ideal of all $\JJ$-classes below $J$, which is given by
$S^1 J S^1 \setminus J$.  This allows us to make the following definition.

\begin{definition}
  \label{def:princfact}
  \index{principal factor}
  \nomenclature[-]{$\pf{\phantom{D}}$}{Principal factor}
  Let $S$ be a semigroup, and let $J$ be a $\JJ$-class of $S$.  The
  \textbf{principal factor} of $J$ is denoted by $\pf{J}$, and defined as
  follows:
  \begin{itemize}
  \item $\pf{J} = J$ if $J$ is the minimal ideal;
  \item $\pf{J} = S^1 J S^1 / (S^1 J S^1 \setminus J)$ otherwise.
  \end{itemize}
\end{definition}

We will identify a principal factor $\pf{J}$ as the set $J \cup \{0\}$, with
multiplication $\circ$ defined by
$$a \circ b = \left\{
  \begin{array}{l l}
    ab & \text{if~} a,b \in J; \\
    0 & \text{otherwise}.
  \end{array}\right.$$
In the case that $J$ is the minimal ideal of $S$, we do not need to append a
zero.

Since $\pf{J}$ is composed of a single $\JJ$-class, possibly with a zero appended,
it is a simple or 0-simple semigroup.  Hence, if $S$ is finite, we may also
identify $\pf{J}$ with a Rees matrix semigroup $\mathcal{M}[G;I,\Lambda;P]$ or Rees
0-matrix semigroup $\mathcal{M}^0[G;I,\Lambda;P]$, by the Rees Theorem (Theorem
\ref{thm:rees}).  This will help us to classify its congruences later.

\subsection{Full transformation monoid $\T_n$}
\label{sec:princfact-tn}

Now we will consider the principal factors of an important monoid, the full
transformation monoid $\T_n$.  Recall that $\T_n$ is the semigroup consisting of
all transformations on the set $\{1, \dots, n\}$, for some $n \in \mathbb{N}$
(Definition \ref{def:tn}).  In order to describe the principal factors of
$\T_n$, we must first consider its Green's relations, as follows.

\begin{proposition}
  \label{prop:tn-greens}
  Let $n \in \mathbb{N}$, and let $\T_n$ be the full transformation monoid of
  degree $n$.  For two mappings $\alpha, \beta \in \T_n$, the following hold:
  \begin{itemize}
  \item $\alpha \LL \beta$ if and only if $\im \alpha = \im \beta$,
  \item $\alpha \RR \beta$ if and only if $\ker \alpha = \ker \beta$,
  \item $\alpha \DD \beta$ if and only if $\rank \alpha = \rank \beta$.
  \end{itemize}
  % TODO: prove this myself
\end{proposition}

The last part of the above proposition allows us to name the semigroup's
$\DD$-classes $D_1^n, D_2^n, \dots, D_n^n$, where each $D_k^n$ is the
$\DD$-class of $\T_n$ consisting of transformations with rank $k$.  Then the
usual partial ordering of $\JJ$-classes (which in a finite semigroup are the
same as $\DD$-classes) gives $D_1^n < D_2^n < \dots < D_n^n$.

Inside a given $\DD$-class $D_k^n$, elements are divided into $\LL$-classes
according to their image set; since all elements have rank $k$, their images
must have size $k$, and so there are $\binom{n}{k}$ $\LL$-classes in total.
Similarly, the elements of $D_k^n$ are divided into $\RR$-classes according to
their kernel; the possible kernels are all $k$-partitions of an $n$-set, so the
total number of $\RR$-classes is given by the Stirling number of the second
kind, $S(n,k)$ \citeoeis{A008277}.

Each $\HH$-class in $D_k^n$ is the intersection of an $\LL$-class and an
$\RR$-class, so each one corresponds to an image-kernel pair (hence we will talk
about the \textit{image and kernel of an $\HH$-class}).  For a given kernel with
$k$ classes and a given image with $k$ elements, there are $k!$ different ways
to assign image elements to kernel classes, simply the permutations of the set
$\{1, \dots, k\}$.  Hence there are $k!$ elements in each $\HH$-class.

\subsubsection{Group $\HH$-classes of $\T_n$}
To understand the principal factor corresponding to a $\DD$-class $D_k^n$, we need
to understand which of its $\HH$-classes are groups and whcih are not.  To
determine which $\HH$-classes are groups, we recall that in any semigroup an
$\HH$-class H is a group if and only if it contains an idempotent (an element
$\alpha \in H$ such that $\alpha \alpha = \alpha$).  A transformation
$\alpha \in \T_n$ is an idempotent if and only if each point in its image is
mapped by $\alpha$ to itself, i.e.
$$i \alpha = i \quad (\forall i \in \im \alpha).$$
Given an image and a kernel, we can choose a transformation with this condition
if and only if no pair of points in the image are in the same kernel-class---
that is, each image point is in a different kernel-class.  Hence an $\HH$-class
of $D_k^n$ is a group if and only if its image contains one point from each class
of its kernel (i.e.~its image is a \textit{cross-section} of its kernel).
\index{cross-section}

\begin{lemma}
  \label{lem:dk-hclasses}
  Let $k,n \in \mathbb{N}$ with $k \leq n$, and let $D_k^n$ be the $\DD$-class
  of $\T_n$ consisting of the elements of rank $k$.  The following hold:
  \begin{enumerate}[\rm(1)]
  \item For any two distinct $\RR$-classes $R_1$ and $R_2$ of $D_k^n$ there is an
    $\LL$-class $L$ such that $L \cap R_1$ is a group $\HH$-class, but
    $L \cap R_2$ is not.
  \item If $k > 1$, then for any two distinct $\LL$-classes $L_1$ and $L_2$ of
    $D_k^n$ there is an $\RR$-class $R$ such that $L_1 \cap R$ is a group
    $\HH$-class, but $L_2 \cap R$ is not.
  \end{enumerate}
  \begin{proof}
    For (1), let $R_1$ and $R_2$ be distinct $\RR$-classes of $D_k^n$.  These two
    classes correspond to distinct kernels $P_1$ and $P_2$, each partitioning
    $\{1, \dots, n\}$ into $k$ classes.  If $k=n$ then there is only one
    possible partition, $\big\{\{1\}, \dots, \{n\}\big\}$, and so $R_1$ and
    $R_2$ cannot be distinct.  If $k<n$ then there must be a pair of elements
    $i,j \in \{1,\dots, n\}$ which are in different classes of $P_1$ but the
    same class of $P_2$.  Let $X$ be a $k$-set containing one element from each
    class of $P_1$, including $i$ and $j$ -- clearly it is a cross-section of
    $P_1$.  But now $X$ contains two elements from one class of $P_2$, so it is
    not a cross-section of $P_2$.  Hence, if $L$ is the $\LL$-class
    corresponding to image $X$, $L \cap R_1$ is a group $\HH$-class but
    $L \cap R_2$ is not a group $\HH$-class.

    For (2), let $L_1$ and $L_2$ be distinct $\LL$-classes of $D_k^n$, with
    $1 < k \leq n$.  These two classes correspond to distinct images of size $k$
    in $\{1 \dots n\}$; let us call these images $I_1$ and $I_2$ respectively.
    Without loss of generality, let $I_1 = \{1, 2, \dots, k\}$.  Since
    $I_1 \neq I_2$, there must be an element $i \in \{1 \dots k\}$ not in $I_2$.
    Now consider the $k$-partition $P$ which puts each element from
    $\{1 \dots k\}$ in a class on its own, apart from one element
    $j \in \{1 \dots k\}$ not equal to $i$, which is in a class with all the
    elements $\{k+1 \dots n\}$ (choosing $j \neq i$ requires $k > 1$).  Now
    $I_1$ is a cross-section of $P$, having precisely one element from each
    class; but $I_2$ does not have an element from the class $\{i\}$, and so it
    is not a cross-section of $P$.  Let $R$ be the $\RR$-class with kernel $P$,
    and we have that $L_1 \cap R$ is a group $\HH$-class but $L_2 \cap R$ is
    not.
  \end{proof}
\end{lemma}

\subsubsection{Principal factors of $\T_n$}
As mentioned in Section \ref{sec:princfact-def}, any principal factor is either
simple or 0-simple, and so it can be identified with a Rees matrix semigroup or
Rees 0-matrix semigroup.  Hence, for some $k > 1$, let
$\pf{D_k^n} = \mathcal{M}^0[G;I,\Lambda;P]$.  The rows and columns of the matrix
$P$ correspond respectively to the $\LL$-classes and $\RR$-classes of $D_k^n$,
and $G$ is the group isomorphic to each of the group $\HH$-classes of $D_k^n$.
Since the elements of an $\HH$-class here correspond to all the permutations of
its image (all the different ways to assign the $k$ image points to the $k$
classes of the kernel) this group is isomorphic to the symmetric group $S_k$.

To consider the congruences of $\pf{D_k^n}$, we first recognise the universal
congruence $\nabla_{\pf{D_k^n}}$.  All the other congruences are in bijective
correspondence with the linked triples of $\pf{D_k^n}$.  Recall the definition
of a linked triples $(N,\mathcal{S},\mathcal{T})$, from Definition
\ref{def:linked-triple}---that is, a normal subgroup $N \trianglelefteq G$, an
equivalence relation $\mathcal{S}$ on $I$ and an equivalence relation
$\mathcal{T}$ on $\Lambda$, such that the following are satisfied:
\begin{enumerate}[\rm(1)]
\item $\mathcal{S} \subseteq \varepsilon_I$, where
  $\varepsilon_I = \left\{(i,j) \in I \times I\, \middle|\, \forall \lambda \in
    \Lambda: p_{\lambda i}=0 \iff p_{\lambda j}=0 \right\}$;
\item $\mathcal{T} \subseteq \varepsilon_\Lambda$, where
  $\varepsilon_\Lambda = \left\{(\lambda,\mu) \in \Lambda \times \Lambda\,
    \middle|\, \forall i \in I: p_{\lambda i}=0 \iff p_{\mu i}=0 \right\}$;
\item For all $i,j \in I$ and $\lambda, \mu \in \Lambda$ such that
  $p_{\lambda i}, p_{\lambda j}, p_{\mu i}, p_{\mu j} \neq 0$ and either
  $(i,j) \in \mathcal{S}$ or $(\lambda,\mu) \in \mathcal{T}$, we have that
  $q_{\lambda \mu i j} \in N$, where
  $$q_{\lambda \mu i j} = p_{\lambda i} p_{\mu i}^{-1} p_{\mu j} p_{\lambda
    j}^{-1}.$$
\end{enumerate}
We shall first find all the triples which satisfy conditions (1) and (2), and
then we shall show that in this case all of them satisfy condition (3).

First, recall that an element $p_{\lambda i}$ is non-zero if and only if the
corresponding $\HH$-class is a group.  By Lemma \ref{lem:dk-hclasses} we can see
that for any pair of columns $i,j \in I$ there exists a row
$\lambda \in \Lambda$ such that $p_{\lambda i} \neq 0 = p_{\lambda j}$.  Hence
$\varepsilon_I = \Delta_I$.  Similarly, in the limited case that $k>1$, Lemma
\ref{lem:dk-hclasses} gives us that for any pair of rows
$\lambda, \mu \in \Lambda$ there exists a column $i \in I$ such that
$p_{\lambda i} \neq 0 = p_{\mu i}$.  Hence if $k>1$ then we have
$\varepsilon_\Lambda = \Delta_\Lambda$.

\subsubsection{Linked Triples for $k = 1$}
\label{sec:k1}
First let us consider the linked triples of $\pf{D_1^n}$.  Since this
$\DD$-class consists of the transformations with rank $1$, its elements have $n$
possible images,
$$\{1\}, \{2\}, \dots, \{n\}$$
and only one possible kernel,
$$\big\{\{1, \dots, n\}\big\}.$$
Hence the matrix $P$ of $\pf{D_1^n}$ has $n$ rows and $1$ column.  Every element in
$D_1^n$ has the form
$$\begin{pmatrix}
  1 & 2 & \cdots & n \\
  i & i & \cdots & i
\end{pmatrix}$$ for some $i \in \{1, \dots, n\}$, so each element is an
idempotent in its own $\HH$-class.  Hence each $\HH$-class is a group, so the
matrix $P$ has no zeroes, and $\varepsilon_\Lambda = \Lambda \times \Lambda$.
The underlying group $G$ of the Rees 0-matrix semigroup $\pf{D_1^n}$ must be
trivial, since each $\HH$-class contains just one element.

Taking all this information together, we can classify all the triples $(N,
\mathcal{S}, \mathcal{T})$ which satisfy conditions (1) and (2) as follows:
\begin{itemize}
\item $N$ must be a normal subgroup of the trivial group---hence $N = 1$;
\item $\mathcal{S}$ must be a subset of the trivial relation $\Delta_I$---hence
  $\mathcal{S} = \Delta_I$;
\item $\mathcal{T}$ may be any equivalence on $\Lambda$.
\end{itemize}
This gives us all triples of the form $(1,\Delta_I,\mathcal{T})$, where
$\mathcal{T}$ can be any partition of the $n$ rows in $\Lambda$.  The number of
these triples is the Bell number $B_n$.  Now consider condition (3): since the
underlying group of $\pf{D_1^n}$ is trivial, and our chosen normal subgroup $N$ is
also trivial, we have that any four nonzero elements from the matrix $P$ must
multiply together to give the identity $1$, which will always be in $N$.  Hence all the
triples described are \textit{linked}, and there are $B_n$ of them.

\subsubsection{Linked Triples for $k \geq 2$}
\label{sec:k2}
Now let us consider the linked triples of $\pf{D_k^n}$ for $k \geq 2$.  We already
know that $\varepsilon_I = \Delta_I$ and $\varepsilon_\Lambda = \Delta_\Lambda$,
so any triple satisfying conditions (1) and (2) must have the form
$$(N, \Delta_I, \Delta_\Lambda)$$
with freedom only in the choice of a normal subgroup $N$ of $G$.  We may write
this simply as $(N, \Delta, \Delta)$ for brevity.  This
underlying group $G$ is, as stated above, isomorphic to the symmetric group
$S_k$, so $N$ can be chosen to be any normal subgroup of $S_k$.

The only normal subgroups of $S_k$ for $k=3$ and $k \geq 5$ are the trivial group, the
alternating group $A_k$, and the symmetric group $S_k$ itself.  For $k=2$ we
have $1 = A_2 < S_2$, and for $k=4$ alone we must add a fourth normal subgroup,
$K_4 = \langle (1~2)(3~4), (1~3)(2~4) \rangle$.

To see that all these triples also fulfil condition (3) we use the triviality of
the relations $\mathcal{S} = \Delta_I$ and $\mathcal{T} = \Delta_\Lambda$.
Observe that $(i,j) \in \mathcal{S}$ only if $i = j$, and $(\lambda,\mu) \in
\mathcal{T}$ only if $\lambda = \mu$.  In the former case, we have
$$q_{\lambda \mu i j} = p_{\lambda i} (p_{\mu i}^{-1} p_{\mu i}) p_{\lambda
  i}^{-1} = p_{\lambda i} p_{\lambda i}^{-1} = 1 \in N,$$
and in the latter case,
$$q_{\lambda \mu i j} = (p_{\lambda i} p_{\lambda i}^{-1}) (p_{\lambda j}
p_{\lambda j}^{-1}) = 1 \cdot 1 = 1 \in N.$$ Hence condition (3) is fulfilled
and all of the triples described are \textit{linked}.

\subsubsection{Numbers of Congruence Classes}
\label{sec:nrclasses}
The universal congruence $\pf{D_k^n} \times \pf{D_k^n}$ has, by definition, one
congruence class.  Any other congruence on a principal factor has a linked
triple $(N,\mathcal{S},\mathcal{T})$, and we can use this triple to calculate
the number of congruence classes.  Each non-zero class corresponds to a triple
$(Nx,[i]_\mathcal{S},[\lambda]_\mathcal{T})$ consisting of a coset of $N$, a
class of $\mathcal{S}$ and a class of $\mathcal{T}$, as described in
\cite[Theorem 3.2]{mtorpey_pre_msc}.  Hence the total number of classes is equal
to the product of the index $|G:N|$, the number of classes of $\sS$, and the
number of classes of $\tT$, plus $1$ for the universal congruence.

\subsubsection{Summary of Results}
\label{sec:summary}
We can now describe all the congruences of the principal factors $\pf{D_k^n}$ of the
full transformation monoid $\mathcal{T}_n$.  If $(N,\mathcal{S},\mathcal{T})$ is a linked
triple on $\pf{D_k^n}$, then let $[N,\mathcal{S},\mathcal{T}]$ be the non-universal
congruence associated with that triple.
For brevity, let $[N] = [N, \Delta_I, \Delta_\Lambda]$ and
let $h_n = \binom{n}{k} \cdot S(n,k)$, the number of
$\HH$-classes in $D_k^n$.

\begin{theorem}
  \label{thm:dkstar-congs}
  The congruences of $\pf{D_k^n}$ are shown in Table \ref{tab:dkstar-congs}.
  \begin{table}[h]
    \centering
    \renewcommand{\arraystretch}{1.3}
    \begin{tabular}{| r | r | c | r |}
      \hline
      \multicolumn{1}{|c|}{$k$} & \multicolumn{1}{|c|}{\textbf{Congruences of $\pf{D_k^n}$}} & \textbf{Number} & \multicolumn{1}{|c|}{\textbf{Number of classes}} \\
      \hline
      $1$ & $[1, \Delta_I, \mathcal{T}] (\forall \mathcal{T})$ & $B_n$
          & from $1$ to $n$ \\
      $2$ & $[1], [S_2], \nabla$ & $3$ & $2h_2+1, h_2+1, 1$ \\
      $3$ & $[1], [A_3], [S_3], \nabla$ & $4$ & $6h_3+1, 2h_3+1, h_3+1, 1$ \\
      $4$ & $[1], [K_4], [A_4], [S_4], \nabla$ & $5$ & $24h_4+1, 6h_4+1, 2h_4+1, h_4+1, 1$ \\
      $\geq 5$ & $[1], [A_k], [S_k], \nabla$ & $4$ & $k!h_k+1, 2h_k+1, h_k+1, 1$ \\
      \hline
    \end{tabular}
    \caption{Summary of the congruences of the principal factors of $\T_n$}
    \label{tab:dkstar-congs}
  \end{table}
\end{theorem}

We can now summarise the numbers of congruence classes for some small values of
$n$.  Table \ref{tab:dkstar-nrclasses} gives the number of classes of each
congruence on each principal factor $\pf{D_k^n}$ of $\T_n$, for $n$ up to $7$.
Note that for $k=1$ only the set of distinct values has been given, since there
are $B_n+1$ different congruences which must be considered.

\begin{table}[h]
  \centering
  \renewcommand{\arraystretch}{1.3}
  \begin{tabular}{|r|r|r|r|r|r|r|r|r|}
    \hline
    & $n=1$ & $n=2$ & $n=3$ & $n=4$ & $n=5$ \\ \hline
    $k=1$ & $1$ & $1$ to $2$ & $1$ to $3$ & $1$ to $4$ & $1$ to $5$ \\
    $k=2$ & -- & 3, 2, 1 & 19, 10, 1 & 85, 43, 1 & 301, 151, 1 \\
    $k=3$ & -- & -- & 7, 3, 2, 1 & 145, 49, 25, 1 & 1501, 501, 251, 1 \\
    $k=4$ & -- & -- & -- & 25, 7, 3, 2, 1 & 1201, 301, 101, 51, 1 \\
    $k=5$ & -- & -- & -- & -- & 121, 3, 2, 1 \\
    \hline
  \end{tabular}

  \phantom{BLANK}
  
  \begin{tabular}{|r|r|r|}
    \hline
    & $n=6$ & $n=7$ \\ \hline
    $k=1$ & $1$ to $6$ & $1$ to $7$ \\
    $k=2$ & 931, 466, 1 & 2647, 1324, 1 \\
    $k=3$ & 10801, 3601, 1801, 1 & 63211, 21071, 10536, 1 \\
    $k=4$ & 23401, 5851, 1951, 976, 1 & 294001, 73501, 24501, 12251, 1 \\
    $k=5$ & 10801, 181, 91, 1 & 352801, 5881, 2941, 1 \\
    $k=6$ & 721, 3, 2, 1 & 105841, 295, 148, 1 \\
    $k=7$ & -- & 5041, 3, 2, 1 \\
    \hline
  \end{tabular}
  \caption[Congruence classes on principal factors of $\T_n$]{Number of classes
    of the congruences on the principal factors of $\T_n$, for $n$ up to $7$}
  \label{tab:dkstar-nrclasses}
\end{table}

\begin{table}[h]
  \centering
\end{table}

\subsection{Other monoids}



\section{The number of congruences of a semigroup}
\label{sec:nrcongs}

Now that it is possible to compute the congruences of a semigroup, we may be
interested in the number of congruences a given semigroup possesses.  At the
very least, a semigroup $S$ must have congruences $\Delta_S$ and $\nabla_S$,
which are equal if and only if $|S| = 1$; so any semigroup has at least one
congruence.  For an upper bound, consider that a congruence is an equivalence;
the number of equivalences on a set is given by the Bell numbers
\citeoeis{A000110}, so $S$ cannot have more congruences than the Bell number
$B_{|S|}$.  All semigroups have a number of congruences between these two
bounds, but the precise number depends on the structure of the semigroup.

In this section, we consider how many congruences there are on various
semigroups, showing some computational results on small semigroups, as well as
proving some more general results.

\subsection{Congruence-full semigroups}
% TODO: this section needs a full rethink of infinite semigroups

We start by giving the definition of a \textit{congruence-full} semigroup, by
analogy with a \textit{congruence-free} semigroup \cite[\S3.7]{howie}.

\begin{definition}
  A semigroup $S$ is \textbf{congruence-full} if every equivalence relation on
  $S$ is a congruence.
\end{definition}

Since the number of equivalences on a set is given by the sequence of Bell
numbers $(B_n)_{n \in \mathbb{N}}$, we can say that a semigroup is
congruence-full if and only if it has precisely $B_n$ congruences, where $n$ is
the size of the semigroup.
% TODO: semigroup might be infinite!  No such thing as n.

Finite congruence-free semigroups are classified in \cite[3.7]{howie}.
In this section we explore finite congruence-full semigroups, culminating in a
complete classification in Theorem \ref{thm:congruence-full}.  First we need to
build up some knowledge about the Green's relations of congruence-full
semigroups.

\begin{lemma}
  \label{lem:m-is-h-trivial}
  A congruence-full semigroup of size greater than $2$ has $\HH$-trivial
  minimal ideal.
  % TODO: infinite?  Does it have to have a minimal ideal?
  \begin{proof}
    Let $S$ be a congruence-full semigroup with more than $2$ elements, and let
    $M$ be its minimal ideal.  Since $M$ is simple, every $\HH$-class of $M$ is
    a group.  We will proceed by considering possible sizes of $M$'s
    $\HH$-classes, and showing that any size greater than $1$ violates the
    assumption that $S$ is congruence-full.

    Firstly, let $H$ be an $\HH$-class in $M$ with at least $3$ elements.  Let
    $1_H$ be the group identity of $H$, and let $g,h \in H\setminus\{1_H\}$
    with $g \neq h$.  Now let $\sim$ be $(1_H, g)^e$.  Since $g$ is not the
    identity, we know that $gh \neq h$.  Hence we have $1_H \sim g$ but
    $1_H h \nsim gh$, so $\sim$ is not a congruence.  Hence $S$ is not
    congruence-full, a contradiction.

    Instead, let $H$ be an $\HH$-class in $M$ with precisely $2$ elements.
    Since $|S| \geq 3$ we know that $S \setminus H$ is non-empty.  If there
    exists some $x \in S \setminus M$, then let $h \in H \setminus \{1_H x\}$,
    and let $\sim$ be $(x,h)^e$; since $h \neq 1_H x$ we have $x \sim h$ but
    $1_H x \nsim 1_H h$, so $\sim$ is not a congruence.  If on the other hand
    $S \setminus M$ is empty, then $M$ must contain an $\HH$-class other than
    $H$.  Choose some $x \in M \setminus H$ such that $x \LL 1_H$ (if this is
    not possible, we can choose $x$ such that $x \RR 1_H$, and a similar
    argument holds).  Let $h \in H \setminus \{1_H\}$, and let $\sim$ be
    $(x,1_H)^e$.  We have $xh \RR x \nRR h$, so $xh \nRR h$ and in particular
    $xh \neq h$.  Hence $x \sim 1_H$ but $xh \nsim 1_H h$, so $\sim$ is
    not a congruence.  Either of these cases violates the assumption that $S$
    is congruence-full, a contradiction.
  \end{proof}
\end{lemma}

\begin{lemma}
  \label{lem:m-is-l-or-r-trivial}
  A congruence-full semigroup of size greater than $2$ has a minimal ideal which
  is either $\LL$-trivial or $\RR$-trivial.
  \begin{proof}
    Let $S$ be a congruence-full semigroup with more than $2$ elements, with a
    minimal ideal $M$ which is neither $\LL$-trivial nor $\RR$-trivial.

    We know by Lemma \ref{lem:m-is-h-trivial} that $M$ is $\HH$-trivial.
    Since $M$ is simple and $\HH$-trivial, it is a rectangular band.  Let
    $x_{11}, x_{12}, x_{22} \in M$ be pairwise distinct elements with
    $x_{11} \RR x_{12} \LL x_{22}$, and let $\sim$ be the relation
    $(x_{11},x_{22})^e$.  Since $M$ is a rectangular band, we have
    $x_{11}x_{22} = x_{12}$ and $x_{11}x_{11} = x_{11}$.  Hence
    $x_{11} \sim x_{22}$ but $x_{11}x_{11} \nsim x_{11}x_{22}$, and so $\sim$ is
    not a congruence.  This means that $S$ is not congruence-full, a
    contradiction.
  \end{proof}
\end{lemma}

\begin{lemma}
  \label{lem:simple-or-zero-semigroup}
  A finite congruence-full semigroup of size greater than $2$ is either simple
  or a zero semigroup.
  \begin{proof}
    Let $S$ be a congruence-full semigroup with more than $2$ elements, with
    minimal ideal $M$.  Let us assume $S$ is not simple; this means that
    $S \setminus M$ is non-empty.  By Lemma \ref{lem:m-is-l-or-r-trivial}, $M$
    is either $\LL$-trivial or $\RR$-trivial; without loss of generality let us
    assume that $M$ is $\LL$-trivial (a similar argument applies for
    $\RR$-triviality).  We will start by proving that $S$ contains a zero, and
    then we will go on to prove that $S$ is a zero semigroup.
    Firstly, aiming for a contradiction, let us assume that $|M| > 1$.

    If $S \setminus M$ contains an idempotent, call it $x$.  Choose $m,n \in M$
    with $m \neq n$.  Now, either $mx = m$ or $mx \neq m$.  If $mx = m$, then
    let $\sim$ be $(n,x)^e$: since by $\LL$-triviality $mn = n$, and since
    $mx = m$, we have $n \sim x$ but $mn \nsim mx$, so $\sim$ is not a
    congruence, a contradiction.  If on the other hand $mx \neq m$, then let
    $\sim$ be $(m,x)^e$: since $x \neq mx \neq m$ and $xx=x$, we have $m \sim x$
    but $mx \nsim xx$, so $\sim$ is not a congruence, a contradiction.

    If instead, $S \setminus M$ does not contain an idempotent, then there must
    exist some $x \in S \setminus M$ such that $x^2 \in M$.
    \footnote{Does this fact need proof?  Does it hold for infinite semigroups?
    If so, I think we can remove the word ``finite'' everywhere and the final
    theorem is much stronger.}
    Let $m \in M \setminus \{x^2\}$ (this is possible since $|M| > 1$) and let
    $\sim$ be $(m,x)^e$.  Since by $\LL$-triviality $xm = m$, we have $m \sim x$
    but $xm \nsim xx$, so $\sim$ is not a congruence, a contradiction.

    We have now shown that $|M| > 1$ violates the condition that $S$ is
    congruence-full.  Hence the minimal ideal $M$ must contain precisely one
    element, $0$: we have $0x = x0 = 0$ for any $x \in S$, so $0$ is a zero for
    $S$. Next we will show that $S$ is a zero semigroup, i.e.~that $xy = 0$ for
    all $x,y \in S$.  Clearly if $x$ or $y$ is $0$ then $xy = 0$.

    Let $x,y \in S \setminus \{0\}$ with $x \neq y$.  The product $xy$ cannot be
    equal to both $x$ and $y$, so without loss of generality let us assume that
    $xy \neq x$.  Assume, aiming for a contradiction, that $xy \neq 0$.  Let
    $\sim$ be the relation $(x,0)^e$; since $0y = 0$ and $xy \neq x$ we have
    $x \sim 0$ but $xy \nsim 0y$, so $\sim$ is not a congruence, a
    contradiction.  Hence for distinct $x,y \in S$ we have $xy = 0$.

    It only remains to consider whether $x^2=0$ for every $x \in S$.  Let
    $x \in S \setminus \{0\}$ and assume, aiming for a contradiction, that
    $x^2 \neq 0$.  Let $y \in S \setminus \{0,x\}$ (possible since $|S| > 2$)
    and let $\sim$ be $(x,y)^e$; since $xy=0$ but $x^2 \neq 0$, we have
    $x \sim y$ but $xx \nsim xy$, so $\sim$ is not a congruence, a
    contradiction.  Hence $xy = 0$ for all $x,y \in S$, so $S$ is a zero
    semigroup.
  \end{proof}
\end{lemma}

We can now state the main theorem of this section, a classification of all the
finite congruence-full semigroups.

\begin{theorem}
  \label{thm:congruence-full}
  A finite semigroup is congruence-full if and only if it is a zero semigroup, a
  left zero semigroup, a right zero semigroup, or has size less than or equal to
  $2$.
  \begin{proof}
    Let $S$ be a finite congruence-full semigroup of size greater than $2$.  If
    $S$ is not simple, then by Lemma \ref{lem:simple-or-zero-semigroup} it is a
    zero semigroup.  If $S$ is simple, it is equal to its minimal ideal.
    Therefore, by Lemma \ref{lem:m-is-l-or-r-trivial}, $S$ is either
    $\LL$-trivial or $\RR$-trivial.  If it is $\LL$-trivial then $xy=y$ for all
    $x,y \in S$, so it is a right zero semigroup.  If it is $\RR$-trivial then
    $xy=x$ for all $x,y \in S$, so it is a left zero semigroup.

    To prove the converse, we consider zero, left zero, and right zero
    semigroups in turn.  First, let $S$ be a zero semigroup and let $\sim$ be an
    equivalence relation on $S$.  Let $x,y,s,t \in S$ such that $x \sim y$ and
    $s \sim t$.  We have $xs = 0 = yt$, so $xs \sim yt$ and therefore $\sim$ is
    a congruence.  Hence $S$ is congruence-full.

    Alternatively, let $S$ be a left zero semigroup and let $\sim$ be an
    equivalence relation on $S$.  Let $x,y,a \in S$ such that $x \sim y$.  We
    have $ax = a = ay$ and $xa = x \sim y = ya$, so $ax \sim ay$ and
    $xa \sim ya$.  Hence $\sim$ is a congruence, so $S$ is congruence-full.
    A similar argument proves the statement for right zero semigroups.

    % Mention size 2
  \end{proof}
\end{theorem}



\subsection{Semigroups with fewer congruences}

If a semigroup of size $n$ is not congruence-full, it has fewer than $B_n$
congruences.  For some low values of $n$ a semigroup may have $B_n - 1$
congruences, but as $n$ grows the number of congruences seems to be lower.  In
this section we propose a value for the ``second highest'' number of congruences
on a semigroup of size $n$.

\begin{conjecture}
  \label{conj:not-cong-full}
  A semigroup of size $n > 3$ which is not congruence-full has at most
  $2B_{n-1}$ congruences.
\end{conjecture}

This conjecture, which does not yet have a proof, is supported by experimental
investigation.  An exhaustive analysis of all semigroups up to isomorphism and
anti-isomorphism shows that the conjecture holds for $n \leq 7$, and also
reveals a pattern in the semigroups which attain the limit.  This pattern is
stated in the next conjecture.

\begin{conjecture}
  \label{conj:cong-nearfull-7}
  Let $n > 3$.  There are precisely 7 semigroups (up to isomorphism and
  anti-isomorphism) of size $n$ which have $2B_{n-1}$ congruences.
\end{conjecture}

The 7 semigroups are described in the following table, and a diagram is given
for $n = 4$.  In the descriptions, when $\Z_{n-1}$ is a subsemigroup of $S$, the
zero of $\Z_{n-1}$ is called $z$.

\begin{longtable}{| c | >{\centering}m{0.5\linewidth} | m{0.3\linewidth} |}
  \hline
  No. & Diagram of $S$ & Description of $S$
  \\ \hline
  1 & \includesvg{pics/ch-other/lz-plus-id}
  & Left-zero semigroup $\LZ_{n-1}$ with an idempotent $c$ appended.
  There is a distinguished $p \in M$ such that $cx=p$ and $xc=x$ for
  all $x \in M$.
  \\ \hline
  2 & \includesvg{pics/ch-other/lz-plus-nonid}
  & Left-zero semigroup $\LZ_{n-1}$ with a non-idempotent element $c$
  appended.
  Multiplication defined by $cx=c^2$ and $xc=x$ for all $x \in M$.
  \\ \hline
  3 & \includesvg{pics/ch-other/z-plus-zero}
  & Zero semigroup $\Z_{n-1}$ with a zero appended.
  \\ \hline
  4 & \includesvg[scale=0.8]{pics/ch-other/z-plus-id} % TODO: consistent scale
  & Zero semigroup $\Z_{n-1}$ with an idempotent $c$ appended above the minimal
  ideal.
  Multiplication defined by $cx=xc=z$ for all $x \in \Z_{n-1}$.
  \\ \hline
  5 & \includesvg{pics/ch-other/z-plus-m-id}
  & Zero semigroup $\Z_{n-1}$ with an idempotent $c$ appended in the minimal
  ideal.
  Multiplication defined by $cx=c$ and $xc=z$ for all $x \in \Z_{n-1}$.
  \\ \hline
  6 & \includesvg{pics/ch-other/c2-plus-nonids}
  & Zero semigroup $\Z_{n-1}$ with an element $c$ appended in the same
  $\HH$-class as $z$.
  Multiplication defined by $c^2=z$ and $xc=cx=c$ for all $x \in \Z_{n-1}$.
  \\ \hline
  7 & \includesvg{pics/ch-other/c2-plus-nonids}
  & Cyclic group $C_2 = \{\id, g\}$ with $n-2$ elements appended such that
  $xy=\id$, $x \id=\id x=g$ (for all $x,y \in S \setminus C_2$).
  \\ \hline
\end{longtable}

Of the 7 semigroups in the table above, semigroups 1 to 6 contain a copy of
either $\Z_{n-1}$ or $\LZ_{n-1}$ as a subsemigroup.  The one element outside
this subsemigroup, which we will call $c$ (the \textit{child element}) is key to
understanding why these semigroups have $2B_{n-1}$ congruences.  In each
semigroup, there is another element $p$ (the \textit{parent element}) such that
an equivalence $\sim$ is a congruence if and only if $c \sim p$ or $[c]_\sim$ is
a singleton.  In other words, the child has to be alone or with its parent.
Now, since $S \setminus \{c\}$ is congruence-full, we can take any congruence
(any equivalence) $\sim$ on $S \setminus \{c\}$ and extend it to a congruence on
$S$ in two different ways: by including $c$ as a singleton, or by including $c$
in the same congruence class as $p$.  Since there are $B_{n-1}$ choices for
$\sim$, this gives us precisely $2B_{n-1}$ congruences on $S$.

Semigroup 7 is unique in that it does not contain $\Z_{n-1}$ or $\LZ_{n-1}$ as a
subsemigroup.  However, it still fulfils the \textit{child/parent} condition
above, where $c = \id$ and $p = g$.

\subsection{Other findings}
We applied the methods from Chapter \ref{chap:lattice} to a bunch of semigroups,
including the partition monoids and small semigroups described elsewhere in this
thesis.

Looks like most congruences are principal?

Show some pictures of lattices on famous semigroups.


\singlespacing
\bibliography{bibliography}
\bibliographystyle{alpha}

\singlespacing
\printnomenclature
\clearpage
\phantomsection
\printindex
\doublespacing

\end{document}
