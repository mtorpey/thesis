\begin{abstract}
Semigroup theory has its roots in group theory.
Since any group is a semigroup, it might be imagined that groups would be
studied simply as a special case within semigroup theory.
Computational algebra has existed for nearly as long as the theory of
computation itself.
In the same way that semigroup theory borrows results from group theory,
computational semigroup theory often borrows algorithms from computational group
theory.
This thesis deals primarily with the congruences of a semigroup.
The computational theory of semigroup congruences is also a young field.
After introducing some preliminary theory, this thesis is divided into two broad
parts.
Chapter \ref{chap:intro} acts as an introduction to this document, providing the
preliminary knowledge which is required to understand the material in the rest
of the thesis.
Chapter \ref{chap:pairs} presents a new way of computing with congruences
defined by generating pairs.
Chapter \ref{chap:converting} concerns the various ways of representing a
congruence, other than as a set of pairs.
Chapter \ref{chap:lattice} presents an algorithm for computing the entire
congruence lattice of a finite semigroup.
Chapter \ref{chap:motzkin} considers the Motzkin monoid $\Mot_n$, the monoid of
all planar bipartitions of degree $n$ with blocks of size no greater than $2$.
Chapter \ref{chap:other} completes this thesis by presenting some other results
obtained or precipitated by computational experiments in the \Semigroups{}
package.
At the end of this document we provide an index of the various terms that are
used.
  \thispagestyle{plain}
\end{abstract}
