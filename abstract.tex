\begin{abstract}

  Computational semigroup theory is an area of research that is subject to
  growing interest.  The development of semigroup algorithms allows for new
  theoretical results to be discovered, which in turn informs the creation of
  more new algorithms.  Groups have benefitted from this cycle since before the
  invention of electronic computers, and the popularity of computational group
  theory has resulted in a rich and detailed literature.  Computational
  semigroup theory is a less developed field, but recent work has resulted in a
  variety of algorithms, and some important pieces of software such as the
  \Semigroups{} package for \GAP{}.

  Congruences are an important part of semigroup theory, describing a
  semigroup's homomorphic images in the same way as a group's normal subgroups.
  There currently exist few algorithms for computing with semigroup congruences.
  However, a number of results about alternative representations for
  congruences, as well as existing algorithms that can be borrowed from group
  theory, make congruences a fertile area for improvement.  In this thesis, we
  first consider computational techniques that can be applied to the study of
  congruences, and then present some results that have been produced or
  precipitated by applying these techniques to examples.

  After some preliminary theory, we start with a new parallel approach to
  computing with congruences specified by generating pairs.  We then consider
  alternative ways of representing a congruence, using intermediate objects such
  as normal subgroups and linked triples.  We also present an algorithm for
  computing the entire congruence lattice of a finite semigroup.  In the second
  part of the thesis, we classify the congruences of the Motzkin monoid and
  other monoids of bipartitions, as well as the principal factors of the full
  transformation monoid and several related monoids.  Finally, we consider the
  number of congruences a semigroup can have, and examine the congruences on
  semigroups of size up to seven.
  
  \thispagestyle{plain}
\end{abstract}
