\chapter*{Preface}
\addcontentsline{toc}{chapter}{Preface}

Semigroup theory has its roots in group theory.  Groups in some form were used
in order to solve equations as early as the late 1700s by Lagrange; this work
was continued in the early 1800s by Galois, who first used the word
\textit{group} to describe them.  Groups were then studied in various different
contexts -- in geometry, in number theory, and as permutation groups -- for some
time before the various branches of theory were united into one, with von Dyck
inventing the modern abstract definition of a group in 1882 \cite{dyck_1882},
with the study of group theory flourishing since then.  By contrast, semigroup
theory is a relatively young area of study, having been defined only in the
early 1900s and having been studied very little before the 1950s, other than a
few early papers by authors such as Suschkewitch \cite{susch_1928}.

Since any group is a semigroup, it might be imagined that groups would be
studied simply as a special case within semigroup theory.  However, in practice
it turns out that groups are a very important topic within the study of generic
semigroups.  For example, important features of the structure of a semigroup are
determined by that semigroup's maximal subgroups, for example their Green's
relations, or the linked triples on a completely 0-simple semigroup.  As a
result, groups are often a central part of semigroup theory, allowing us to
borrow from the richly developed field of group theory in order to solve
problems for semigroups in general.

Computational algebra has existed for nearly as long as the theory of
computation itself.  For instance, perhaps the earliest published group theory
algorithm is the Dehn algorithm \cite{dehn_1911} for solving the word problem in
certain groups.  The Todd--Coxeter algorithm \cite{todd_coxeter_1936}, which
enumerates the cosets of a subgroup in a group, was certainly designed to be
carried by hand, having been described in the 1930s before the invention of
electronic computers.  When computers did arrive, there was early interest in
using them for group theory problems, with Todd--Coxeter being implemented on an
the EDSAC II in Cambridge as early as 1953 \cite{leech_1963}.  Since then, group
theory has flourished, with a variety of software packages such as
\textsf{Magma} \cite{magma}, \ACE{} \cite{ace}, and particularly \GAP{}
\cite{gap}, which has a variety of packages including algorithms to solve a wide
range of problems.  A lot of material is available on computational group
theory, including dedicated books \cite{sims, cgt}.  By comparison,
computational semigroup theory is much younger and less developed, as is
semigroup theory as a whole.  Computers were used as early as 1953 to classify
semigroups of low order \cite{tamura_1953, froidure_pin}, but the theory of
computing with semigroups developed more slowly than with groups, and no package
emerged for Semigroups on the scale of the group algorithms in \GAP{}.  However,
there has been considerable interest in computational semigroup theory in recent
years, and increasingly there does exist software for computing with semigroups,
such as \Semigroupe{} \cite{semigroupe}, \libsemigroups{} \cite{libsemigroups},
and the \GAP{} packages \Semigroups{} \cite{semigroups}, \smallsemi{}
\cite{smallsemi} and \kbmag{} \cite{kbmag}.

In the same way that semigroup theory borrows results from group theory,
computational semigroup theory often borrows algorithms from computational group
theory.  The Todd--Coxeter and Knuth--Bendix algorithms, for instance, both of
which were originally applied to groups, extend naturally to semigroups, and to
semigroup congruences, as will be discussed in this thesis.  Furthermore, when
we use the maximal subgroups of a semigroup, we can use group algorithms to
compute information about the semigroup.  For example, when computing the
linked triples on a completely simple semigroup, we can use group theory to find
all the normal subgroups of a given maximal subgroup.  In this way,
computational group theory is not just a subset, but a key part, of
computational semigroup theory.

This thesis deals primarily with the congruences of a semigroup.  A semigroup's
congruences describe its homomorphic kernels and images -- that is, the ways in
which the semigroup can be mapped onto another semigroup while preserving the
semigroup operation.  In this way, a semigroup's congruences serve the same
function as a group's normal subgroups, or a ring's two-sided ideals, though the
definition is more abstract and general than either of these.  The congruences
of several famous semigroups have been classified previously -- for example, the
full transformation monoid $\T_n$ by Mal$'$cev \cite{malcev_1952}, the symmetric
inverse monoid $\I_n$ by Liber \cite{liber_1953} and the partial transformation
monoid $\PT_n$ by Shutov \cite{shutov_1988} -- and in recent years a host of
other monoids of partial transformations, restricted variously to elements that
preserve or reverse the order or orientation of the set being acted on, have
also had their congruences classified \cite{fernandes_2000, lisbon_ii,
  lisbon_i}.  Classifying the congruences of other important semigroups is a
major activity in the study of congruences, and will form the majority of Part
\ref{part:results} of this thesis.

Computing with semigroup congruences is also a young field.  Some algorithms for
computing information about a given congruence have existed in the \GAP{}
library for many years, but many of them lack sophistication, being close to
brute-force.  In some cases, we can do little to improve on these algorithms,
since semigroup congruences in general do not have the regular structure of
%TODO

Chapter 1

Chapter 2

Chapter 3

Chapter 4

Chapter 5

Chapter 6

Index, notation, hyperlinks

% 1500 words
